\section{Dinámica}

\begin{enumerate}
\item A un cuerpo se le aplica una fuerza de 400 N y tiene una masa de 24 kg, ¿cuánto vale su aceleración?
%Rta: 16,7 m/s^2

\item A un cuerpo se le aplica una fuerza de 45.000 N y tiene una masa de 15 toneladas, ¿cuánto vale su aceleración?
%Rta: 3 m/s^2

\item A un cuerpo se le aplica una fuerza de 50 N y tiene una masa de 40 g, ¿cuánto vale su aceleración?
%Rta: 1250 m/s^2

\item Un cuerpo se encuentra bajo el efecto de una fuerza de 1650 N. Su aceleración es 13,25 m/s$2$, ¿cuánto vale su masa? Expresar el resultado en kg y en g. %Rta: 124 kg y 124.000 g

\item Un cuerpo se encuentra bajo el efecto de una fuerza de 1.000.000 N. Su aceleración es 2 m/s$^2$, ¿cuánto vale su masa? Expresar el resultado en kg, en g y en toneladas. %Rta: 500.000 kg; 500.000.000 g; 500 toneladas

\item Un cuerpo tiene una masa de 25 kg y está acelerando a 14 m/s$^2$, ¿cuánto vale la fuerza total ejercida al cuerpo? Expresar el resultado en N y en kgf. %Rta: 350N y 35,7 kgf.


\item Una persona tiene una masa de 60 kg y está en la superficie terrestre, ¿cuánto vale la fuerza total ejercida al cuerpo? Expresar el resultado en N y en kgf. %Rta: 588 y 60 kgf.

\end{enumerate}