\section{Dinámica}

\begin{enumerate}
\item A un cuerpo se le aplica una fuerza de 400 N y tiene una masa de 24 kg, ¿cuánto vale su aceleración?
%Rta: 16,7 m/s^2

\item A un cuerpo se le aplica una fuerza de 45.000 N y tiene una masa de 15 toneladas, ¿cuánto vale su aceleración?
%Rta: 3 m/s^2

\item A un cuerpo se le aplica una fuerza de 50 N y tiene una masa de 40 g, ¿cuánto vale su aceleración?
%Rta: 1250 m/s^2

\item Un cuerpo se encuentra bajo el efecto de una fuerza de 1650 N. Su aceleración es 13,25 m/s$2$, ¿cuánto vale su masa? Expresar el resultado en kg y en g. %Rta: 124 kg y 124.000 g

\item Un cuerpo se encuentra bajo el efecto de una fuerza de 1.000.000 N. Su aceleración es 2 m/s$^2$, ¿cuánto vale su masa? Expresar el resultado en kg, en g y en toneladas. %Rta: 500.000 kg; 500.000.000 g; 500 toneladas

\item Un cuerpo tiene una masa de 25 kg y está acelerando a 14 m/s$^2$, ¿cuánto vale la fuerza total ejercida al cuerpo? Expresar el resultado en N y en kgf. %Rta: 350N y 35,7 kgf.

\item Una persona tiene una masa de 60 kg y está en la superficie terrestre, ¿cuánto vale la fuerza total ejercida al cuerpo? Expresar el resultado en N y en kgf. %Rta: 588 y 60 kgf.

\item Una caja de 8 kg está apoyada en el suelo. Hacer su diagrama de cuerpo libre y obtener el valor de cada fuerza.

\item Una caja de 4330 g está apoyada sobre una rampa que forma 30º con el suelo. Hacer su diagrama de cuerpo libre y obtener el valor de cada fuerza.

\item Una caja de 1,2 kg está apoyada sobre una rampa sin rozamiento, que forma un ángulo de 26º con el suelo. La caja está atada a una cuerda horizontal que evita su movimiento. Hacer su diagrama de cuerpo libre y obtener el valor de cada fuerza.

\end{enumerate}

\subsection*{Momento lineal}

\begin{enumerate}
    \item Un auto de 2000 kg, inicialmente quieto recibe un cañonazo. La bala de cañón tienen una masa de 1.500g. Luego del impacto, el auto empieza a moverse a una velocidad de 3 km por hora. Calcular la velocidad de la bala justo antes del impacto.

    \item Un tren de 120 toneladas que va a 50 km/h choca de frente contra otro tren de 80 toneladas que va a 40 km/h. Siendo que después del choque ambos trenes quedan pegados, calcular a qué velocidad se moverán.

    \item Una bala de 50g de un rifle tarda 0,05 segundos en salir del cañón. Sabiendo que sale del cañón con una velocidad de 800km/h, calcular el impulso ejercido por la pólvora.
\end{enumerate}
