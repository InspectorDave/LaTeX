\section{Trigonometría}

\subsection*{Teorema de Pitágoras}

\hfil
$A^2 + B^2 = C^2$
\hfil
SOHCAHTOA
\hfil

\subsection*{Identidades básicas}

\begin{itemize}
\item $\tan(x) = \dfrac{\sin(x)}{\cos(x)}$
\item $\sec = \dfrac{1}{\cos(x)}$
\item $\csc = \dfrac{1}{\sin(x)}$
\item $\cot = \dfrac{1}{\tan(x)} = \dfrac{\cos(x)}{\sin(x)}$
\item $\sin\left(x + \frac{\pi}{2}\right) = \cos(x)$
\end{itemize}

\subsection*{Identidades importantes}

\begin{itemize}
    \item \makebox[8cm][l]{
    $\sin^2(x) + \cos^2(x) = 1$
    } La más importante de todas

    \item \makebox[8cm][l]{
    $\sin(-x) = -\sin(x)$
    } Imparidad del seno

    \item \makebox[8cm][l]{
    $\cos(x) = \cos(-x)$
    } Paridad del coseno

    \item \makebox[8cm][l]{
    $\sin(x+y) = \sin(x)\cdot\cos(y) + \cos(x)\cdot\sin(y)$
    } Seno de una suma
    
    \item \makebox[8cm][l]{
    $\cos(x+y) = \cos(x)\cdot\cos(y) - \sin(x)\cdot\sin(y)$
    } Coseno de una suma
\end{itemize}


\subsection*{Teorema del seno}

En cualquier triángulo se cumplen las siguientes relaciones, siendo $L_{\text{o1}}$ el lado opuesto a $\alpha$, etc.

$$
\dfrac{\sin(\alpha)}{L_{\text{o1}}} = 
\dfrac{\sin(\beta)}{L_{\text{o2}}} = 
\dfrac{\sin(\gamma)}{L_{\text{o3}}}
$$