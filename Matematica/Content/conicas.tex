\section{Cónicas}


\begin{description}
    \item [Parábola]\hfill

    La parábola se define geométricamente como los puntos que están a la misma distancia de un punto $F$ (foco) y de una recta (directriz).

    Su ecuación canónica es: $$2\cdot p \cdot (y-y_0) = (x-x_0)^2$$
    
    donde $p$ es la distancia desde el foco hasta la directriz. El vértice estará en el punto medio entre el foco y la directriz. En parábolas $e=1$.

    \item [Elipse]\hfill

    La elipse se define geométricamente como los puntos cuya suma de las distancias a dos puntos (focos) es constante. 

    Su ecuación canónica es:
    $$\left(\dfrac{x-x_0}{a}\right)^2 + \left(\dfrac{y-y_0}{b}\right)^2 = 1$$

    Sus focos están a una distancia $c^2 = a^2 - b^2$ del centro.

    En elipses la excentricidad sigue siendo $e=c/a$, además $e<1$. Para averiguar dónde está la directriz con respecto al centro se hace $d=\dfrac{a^2}{c}$. Tener en cuenta que si el semieje mayor es el vertical, en vez de $a$ sería $b$.

    \item [Hipérbola]\hfill

    La hipérbola se define geométricamente como los puntos cuya diferencia de las distancias a dos puntos (focos) es constante (en valor absoluto). 

    Su ecuación canónica es:
    $$\left(\dfrac{x-x_0}{a}\right)^2 - \left(\dfrac{y-y_0}{b}\right)^2 = 1$$

    Sus focos están a una distancia $c^2 = a^2 + b^2$ del centro. La excentricidad es la relación que hay entre la distancia de un foco con su directriz y es constante. Es $e=c/a$, y se cumple la siguiente relación para cualquier punto de la elipse: $d(P,F)/d(P,d)=e$. En hipérbolas $e>1$,

    La distancia de la directriz con respecto al centro es $d=\dfrac{a^2}{c}$.

    \item [Recurso de geogebra que está piola]\hfill

    \url{https://www.geogebra.org/m/qFWgGg2g}
\end{description}