\section{Derivadas}


\begin{multicols}{2}
\subsection*{Por definición}

$$f'(x) = \lim\limits_{h\rightarrow 0}\dfrac{f(x+h) - f(x)}{h}$$


\subsection*{Recta tangente}

La recta tangente es una función lineal, por lo tanto cumplirá $t(x) = m\cdot x + b$.

\noindent
Su pendiente en un punto es la derivada en ese punto:

\hfil$m=f'(x_0)$.\hfil

\noindent
Para obtener su ordenada al origen reemplazar en $t(x)$ con los puntos de las función y la pendiente obtenida, es decir:

\hfil$ y_0 = m \cdot x_0 + b $\hfil

\noindent
Despejar $b$ y listo!

\subsection*{Polinomio de Taylor}

Sirve para aproximar funciones con una nueva:

$$\sum \limits_{n=0}^\infty \frac {{f^{(n)}} (a)}{n!} (x-a)^n$$
\vfill


\subsection*{Tabla de derivadas}
{
\renewcommand{\arraystretch}{2}
\begin{table}[H]
\centering
\begin{tabular}{|c|c|}
\hline
\rowcolor[HTML]{F0F0F0}
$f(x)$ & $f'(x)$ \\ \hline
$k$ & 0 \\ \hline
$x$ & 1 \\ \hline
$x^k$ & $k\cdot x^{k-1}$ \\ \hline 
$e^x$ & $e^x$ \\ \hline
$a^x$ & $a^x \cdot \ln(a)$ \\ \hline
$\ln(x)$ & $\dfrac{1}{x}$ \\ 
& \\[-9mm] \hline
$\log_B(x)$ & $\dfrac{1}{x}\cdot \log_B(e)$ \\
& \\[-9mm] \hline
$\sin(x)$ & $\cos(x)$ \\ \hline
$\cos(x)$ & $-\sin(x)$ \\ \hline
$\tan(x)$ & $\dfrac{1}{\cos^2(x)}$ \\
& \\[-9mm] \hline
$f(x)\cdot g(x)$ & $f'(x) \cdot g(x) + f(x) \cdot g'(x)$ \\ \hline
$\dfrac{f(x)}{g(x)}$ & $\dfrac{f'(x) \cdot g(x) - f(x) \cdot g'(x)}{g^2(x)}$ \\
& \\[-9mm] \hline
$f\,\big(g(x)\big)$ & $f'\big(g(x)\big) \cdot f'(x) $ \\ \hline
\end{tabular}
\end{table}
}
\end{multicols}