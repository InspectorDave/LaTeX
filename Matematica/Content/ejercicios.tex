\subsection*{Ejercicios:}

    \begin{enumerate}
    
    \item Se tienen 1000 g de agua salada, se sabe que hay 50 g de sal. Calcular el $\dfrac{\%\text{m}}{\text{m}}$ de la solución.
    
    \item Se tienen 2 kg de agua, al cual se le agregan 0,5 kg de azúcar. 
    
    \textbf{a)} Calcular la masa de la solución.
    
    \textbf{b)} Calcular el $\dfrac{\%\text{m}}{\text{m}}$ de la solución. 
    
    \item Se tienen 80 g de nitrato de sodio disueltos en 1 kg de agua. Calcular el $\dfrac{\%\text{m}}{\text{v}}$ de la solución.
    
    \item Se tienen 2 kg de nitrato de potasio disuelto en 10 l de glicerol. 
    
    \textbf{a)} Nombrar cuál es la solución, cuál el solvente y cuál el soluto.

    \textbf{b)} Calcular el $\dfrac{\%\text{m}}{\text{v}}$ de la solución.
    
    \item Se tienen 8 l de agua y 2 l de alcohol etílico. 
    
    \textbf{a)} Nombrar cuál es la solución, cuál el solvente y cuál el soluto; además decir cuál es el volumen de cada uno.

    \textbf{b)} Calcular el $\dfrac{\%\text{v}}{\text{v}}$ de la solución.
    
    \item Se tienen 8 l de agua y 12 l de alcohol metílico. 
    
    \textbf{a)} Nombrar cuál es la solución, cuál el solvente y cuál el soluto; además decir cuál es el volumen de cada uno.

    \textbf{b)} Calcular el $\dfrac{\%\text{v}}{\text{v}}$ de la solución.
    
    \end{enumerate}
    
    \hrule