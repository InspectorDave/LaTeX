\section{Análisis de funciones}


\subsection*{Conjunto de positividad/negatividad ($C^+/C^-$)}

Hay que analizar los intervalos entre los puntos críticos (infinitos, raíces, discontinuidades).

{\renewcommand{\arraystretch}{1.25}
\begin{table}[H]
\centering
\begin{tabular}{|c|c|c|c|c|c|}
\hline
& 
\multicolumn{1}{c|}{$\left(-\infty ; -\dfrac{5}{2}\right)$} & 
\multicolumn{1}{c|}{$-\dfrac{5}{2}$} & 
\multicolumn{1}{c|}{$\left(-\dfrac{5}{2}; \dfrac{7}{2}\right)$} & 
$\dfrac{7}{2}$ & 
$\left(\dfrac{7}{2};\infty\right)$ \\ \hline
$f(x)$&    $-$                 &    0                   &         $+$              & \O & $+$ \\ \hline
\end{tabular}
\end{table}
}


\subsection*{Intervalos de crecimiento/decrecimiento ($\text{I}^\nearrow/ \text{I}^\searrow$)}

Hay que analizar los intervalos entre los puntos críticos de la derivada (en la función, estos pueden llegar a ser \textbf{máximos} o \textbf{mínimos}). El conjunto de positividad de la derivada es el intervalo de crecimiento de la función, el conjunto de negatividad de la derivada es el intervalo de decrecimiento de la función.

{\renewcommand{\arraystretch}{1.25}
\begin{table}[H]
\centering
\begin{tabular}{|c|c|c|c|c|c|c|c|}
\hline
&
\multicolumn{1}{c|}{$\left(-\infty ; -\dfrac{\sqrt{3}}{2}\right)$} & 
\multicolumn{1}{c|}{$-\dfrac{\sqrt{3}}{2}$} & 
\multicolumn{1}{c|}{$\left(-\dfrac{\sqrt{3}}{2}; \dfrac{\sqrt{3}}{2}\right)$} & 
$\dfrac{\sqrt{3}}{2}$ & 
$\left(\dfrac{\sqrt{3}}{2};8\right)$ & 8 & $(8; \infty)$\\ \hline
$f'(x)$ & $+$ & 0 & - & 0 & + & 0& +\\ \hline
$f(x)$&  $\uparrow$ & max & $\downarrow$ & min & $\uparrow$ & $\text{nada}$ & $\uparrow$\\ \hline
\end{tabular}
\end{table}
}


\subsection*{Concavidad/convexidad ($\text{I}^\cap/\text{I}^\cup$)}

Para analizar la concavidad, hay que analizar los intervalos entre los puntos críticos de la segunda derivada (en la función estos son los puntos de inflexión).

En el conjunto de positividad de la segunda derivada la función será convexa (también llamada hacia arriba o $\cup$) y en el de negatividad será cóncava (también llamada hacia abajo o $\cap$).

{\renewcommand{\arraystretch}{1.25}
\begin{table}[H]
\centering
\begin{tabular}{|c|c|c|c|c|c|c|c|}
\hline
&
\multicolumn{1}{c|}{$(-\infty ; -6)$} & 
\multicolumn{1}{c|}{$-6$} & 
\multicolumn{1}{c|}{$(-6 ; 2)$} & 
$2$ & 
$(2; 10)$ & $10$ & $(10; \infty)$\\ \hline
$f''(x)$ & $+$ & 0 & - & 0 & + & 0 & 0\\ \hline
$f(x)$ &  $\cup$ & PI & $\cap$ & PI & $\cup$ & nada & nada\\ \hline
\end{tabular}
\end{table}
}


\subsection*{Asíntotas horizontales}

Pueden ocurrir únicamente cuando $x\rightarrow \infty$ o $x\rightarrow -\infty$. Si el límite en esos puntos es un número, entonces habrá una asíntota horizontal en ese número.


\subsection*{Asíntotas verticales}

Ocurren cuando el límite por izquierda o derecha para un valor del codominio es $\infty$ o $-\infty$. En general son para puntos donde hay divisiones por cero, rectas tangentes o logaritmos.

\newpage
\subsection*{Asíntotas oblicuas}

Para saber si existen se tiene que dar que la función tiende a acercarse a una recta oblicua en infinito o menos infinito. Para obtener su fórmula ``$m\cdot x + b$'' se hace:

\skipline
\hfil$m = \lim\limits_{x\rightarrow \pm\infty} \dfrac{f(x)}{x}$\hfil$b = \lim\limits_{x\rightarrow \pm \infty} f(x) - m\cdot x$\hfil

\skipline
\noindent
Tener en cuenta que si no existe una asíntota oblicua $m$ dará $\pm\infty$ o 0 (asíntota horizontal).

