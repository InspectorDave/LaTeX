\section{
\texorpdfstring{Números Complejos $\mathbb{C}$}
{Números Complejos}
}

La base de todo esto es que $\sqrt{-1}=i$

\subsubsection*{Algunos procedimientos frecuentes}

\begin{description}

\item[]\underline{Potencias de $i$:}\hfil

Teniendo $i^k$ , se analiza la operación ``$k$ mod 4'' o ``$k\%4$''. Si sobra 0 es $1$; si sobra 1 es $i$, si sobra 2 es $-1$, si sobra 3 es $-i$ .

\item[]\underline{Raíces:}\hfil

    Al resolver una raíz del estilo $\sqrt{u+iv}$, plantear que ``$u + iv = (a+bi)^2 = a^2 - b^2 + i \cdot 2 \cdot a \cdot b $''.

    Si no resolver la raíz de la forma exponencial y listo.

\item[]\underline{Complejos en denominador:}\hfil

    Multiplicar numerador y denominador por el complemento del denominador, queda una diferencia de cuadrados y finalmente en denominador queda real.
\end{description}


\subsubsection*{Operaciones de números complejos}

\begin{description}
\item[]\underline{Módulo:}\hfil

\hfil$|z|=r=\sqrt{a^2+b^2}$\hfil

\item[]\underline{Argumento:}\hfil

Es el ángulo que forma el número complejo con respecto al eje $x$, va entre $0$ y $2\pi$.

\hfil$\arg (z) = \theta = \arctan(b/a)$\hfil

\end{description}


\subsubsection*{Formas de escribir un número complejo}

\begin{multicols}{3}
\centering

\underline{Binómica}

$z=a+ib$

\underline{Exponencial}

$z=r\cdot e^{i\theta}$

\underline{Trigonométrica}

$z=r \cdot \big(\cos \left(\theta\right) + i\sin\left(\theta\right)\big)$

\end{multicols}
