\section{Cuadráticas}


\subsection{Resolvente}


$$\dfrac{-b \pm \sqrt{b^2 - 4\cdot a \cdot c}}{2\cdot a} = 0$$


\subsection{Formas de función cuadrática}


\subsubsection*{\makebox[4cm][l]{Forma polinómica:}
$f(x)=a\cdot x^2 + b \cdot x + c$}

Para hallar las raíces con esta forma utilizar la resolvente. Sabiendo las raíces se puede averiguar el vértice. 

\subsubsection*{\makebox[4cm][l]{Forma canónica:}
$f(x)= a\cdot(x-x_v)^2 + y_v$}

En esta el vértice de la parábola está en $(x_v, y_v)$.

\subsubsection*{\makebox[4cm][l]{Forma factorizada:}
$f(x)=a\cdot (x-x_1) \cdot (x-x_2)$}

Da información de las raíces, que están en $x_1$ y $x_2$.


\section{Potenciación}

\subsection*{Propiedades de cuadrados}
\begin{itemize}
\item
\makebox[6cm][l]{$(A+B)^2 = A^2 + 2AB + B^2$}
Cuadrado de un binomio (suma)

\item
\makebox[6cm][l]{$(A-B)^2 = A^2 - 2AB + B^2$} 
Cuadrado de un binomio (resta)

\item
\makebox[6cm][l]{$(A+B)\cdot (A-B) = A^2 - B^2$} Diferencia de cuadrados
\end{itemize}


\subsection*{Propiedades generales}

\begin{itemize}
\item
\makebox[6cm][l]{$A^0 = 1,\;\;\;\;\text{siendo } A \neq 0$}
Elevar a la cero

\item
\makebox[6cm][l]{$A^C\cdot A^D = A^{C+D}$}
Producto entre potencias

\item
\makebox[6cm][l]{$\dfrac{A^C}{A^D} = A^{C-D}$}
División entre potencias

\item
\makebox[6cm][l]{$(A\cdot B)^C = A^C\cdot B^C$}
Distributiva para multiplicación de potencias

\item
\makebox[6cm][l]{$\left(\dfrac{A}{B}\right)^C =\dfrac{A^C}{B^C}$}
Distributiva para división de potencias

\item
\makebox[6cm][l]{$A^{-C}=\dfrac{1}{A^C}$}
Potencias negativas

\item
\makebox[6cm][l]{$\sqrt[C]{A}=A^{\frac{1}{C}}$}
Potencias fraccionarias

\item
\makebox[6cm][l]{$\left(A^C\right)^D = A^{C\cdot D}$}
Potencia de potencia
\end{itemize}