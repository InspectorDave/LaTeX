\section{Logaritmos}

Hacer la operación logaritmo responde a la pregunta ``a cuánto tengo que elevar la base $B$ para obtener el argumento $x$'', siendo $\log_B(x)$. Algunas bases son tan conocidas que ni se ponen: ``$\log(x) = \log_{10}(x)$'' y ``$\ln(x) = \log_{e}(x)$''.

\subsection*{Propiedades }
\begin{itemize}
\item \makebox[7cm][l]{$\log_B(1)=0$}
\item \makebox[7cm][l]{$\log _B(x) + \log_B(y)=\log_B(x\cdot y)$}
\item \makebox[7cm][l]{$\log _B(x) - \log_B(y)=\log_B(x / y)$}
\item \makebox[7cm][l]{$\log _B ( x^n)=n\cdot \log_B(x)$}
\item \makebox[6cm][l]{$\log _B (B^x)=B^{\log_B(x)}=x$}
Composición con su inversa
\item \makebox[6cm][l]{$\log_B(A) = \dfrac{\log_C(A)}
{\log_C(B)}$}
Cambio de base
\end{itemize}


