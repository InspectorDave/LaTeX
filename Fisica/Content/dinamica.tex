\section{Dinámica}
    
\subsection*{Segunda ley de Newton}

\hfil$F=m\cdot a$\hfil


\subsection*{Fuerza de rozamiento}

\hfil$F_r = {\mu} \cdot N$\hfil


\subsection*{Campo gravitatorio}

\hfil$F = \dfrac{G\cdot m_1 \cdot m_2}{d^2}$\hfil

La constante de gravitación universal es $G = 6,67\times 10^{-11}\;\dfrac{\text{N}\cdot \text{m}^2}{\text{kg}^2}$


\subsection*{Cantidad de movimiento e Impulso}

\hfil
$p = m\cdot v$
\hfil
$J=\Delta p = F \cdot \Delta t$
\hfil

\subsection*{Fuerzas paralelas y colineales}

Regla de Stevin: $F_1 \cdot d_1 = F_2 \cdot d_2$. La distancia entre ambas fuerzas es $d$, y $d_n$ es la distancia hasta la posición de la resultante. Las distancias y las fuerzas se ponen con signo positivo, siempre. El valor de la fuerza resultante es la suma de ambas. Si ambas tienen mismo sentido, la resultante estará entre ambas; si tienen sentido opuesto la resultante estará por fuera de ambas.