\section{Cinemática}

\subsection{MRU}

\vspace{-\baselineskip}
\begin{align*}
a &= 0\\
v &= \text{cte} = \dfrac{x_1 - x_2}{t_1 - t_2}\\
x(t) &= x_0 + v \cdot t = \text{área bajo la curva de v(t)}
%%$$E(t)= E_0 + v \cdot (t_f - t_0)$$
\end{align*}


\subsection{MRUV}

\vspace{-\baselineskip}
\begin{align*}
a &= \text{cte} = \dfrac{v_1 - v_2}{t_1 - t_2}\\
v(t) &= v_0 + a\cdot t\\
x(t) &= x_0 + v_0\cdot t + \frac{1}{2}\cdot a \cdot t^2 = \text{área bajo la curva de v(t)}    
\end{align*}


\subsection{MCU}

Movimiento Circular Uniforme, es un movimiento en el cual un objeto describe una trayectoria circular, en el cual el tiempo que tarda en dar una vuelta es constante (no hay aceleración tangencial).

Se trabajará con las siguientes magnitudes: el período $T$, es cuánto se tarda en dar una vuelta. La frecuencia $f$ es cuántas vueltas se dan en un determinado tiempo. la velocidad angular $\omega$ es cuántos radianes se recorren por unidad de tiempo. La velocidad tangencial $v$ es cuántos metros se recorren por unidad de tiempo (es análoga a la velocidad de MRU o MRUV). La aceleración centrípeta $a_c$ es la aceleración que se encarga de modificar la dirección de la velocidad, pero no su módulo.

$$ T = \dfrac{1}{f}
\hspace{1cm}
v = \omega\cdot r
\hspace{1cm}
\omega = 2\pi \cdot f
\hspace{1cm}
a_c = \dfrac{v^2}{r}=\omega^2\cdot r
\hspace{1cm}
\theta (t) = \theta _0 + \omega \cdot t
$$


\subsection{MCUV}

Movimiento circular uniformemente variado. Además de una aceleración centrípeta, hay una aceleración angular ($\alpha$) con su aceleración tangencial ($a_t$) asociada constantes.

\hfil
$\theta(t) = \theta_0 + \omega_0 \cdot t + \dfrac{1}{2} \cdot \alpha \cdot t^2$
\hfil
$\omega(t) = \omega_0 + \alpha \cdot t$
\hfil
$a_t = \alpha \cdot r$
\hfil
