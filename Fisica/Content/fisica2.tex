\section{Física 2}

\subsection{Asociación de resistores}

\begin{multicols}{2}

En serie:
$$R_{eq} = R_1 + R_2 + \cdots R_N$$

En paralelo:
$$R_{eq} = \dfrac{1}{\dfrac{1}{R_1} + \dfrac{1}{R_2} + \cdots \dfrac{1}{R_N}}$$
\end{multicols}

\subsection{Capacidad}

\hfil
$C = \dfrac{q}{V}$
\hfil
$E = \dfrac{1}{2}\cdot q \cdot V = \dfrac{1}{2} C \cdot V^2 = \dfrac{q^2}{2C}$
\hfil

\vspace{\baselineskip}
\noindent
\underline{Asociación en serie:}

Al asociar capacitores en serie, la carga total será igual a la carga en cada capacitor. En base a esto se pueden calcular las tensiones en cada uno.

\vspace{\baselineskip}
\noindent
\underline{Asociación en paralelo:}

La diferencia de tensión en cada uno es igual, en base a la capacidad obtener las cargas.

\begin{multicols}{2}
    \subsubsection*{Carga de capacitor}    
    \hfil
    $V = V_{\max} \cdot \left(1-e^{-\frac{t}{\tau}} \right)$
    \hfil

    \subsubsection*{Descarga de capacitor}
    \hfil
    $V = V_{\max} \cdot e^{-\frac{t}{\tau}}$
    \hfil
\end{multicols}
