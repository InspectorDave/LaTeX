\section{Física 2}

\subsection{Capacidad}

\hfil
$C = \dfrac{q}{V}$
\hfil
$E = \dfrac{1}{2}\cdot q \cdot V = \dfrac{1}{2} C \cdot V^2 = \dfrac{q^2}{2C}$
\hfil

\vspace{\baselineskip}
\noindent
\underline{Asociación en serie:}

Al asociar capacitores en serie, la carga total será igual a la carga en cada capacitor. En base a esto se pueden calcular las tensiones en cada uno.

\vspace{\baselineskip}
\noindent
\underline{Asociación en paralelo:}

La diferencia de tensión en cada uno es igual, en base a la capacidad obtener las cargas.
