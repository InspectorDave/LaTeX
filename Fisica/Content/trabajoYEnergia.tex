\section{Trabajo y energía}

La energía de un cuerpo es la capacidad de realizar trabajo que tiene. Al igual que el trabajo, se mide en Joules $J$.


\subsection*{Trabajo}
    
Suele representarse como $W$ (de work) o con $T$. También suele aparecer como $L$. Sus unidades son Joules $J=N\cdot m$. El trabajo que realiza un cuerpo al desplazarse mientras se le aplica una fuerza es igual a la fuerza aplicada en dirección del desplazamiento multiplicado por la distancia ($d$) que recorrió bajo la acción de esa fuerza:

$$W=F\cdot d$$

En caso que la fuerza sea aplicada en la dirección del desplazamiento, el trabajo es positivo, sino es negativo.


\subsection*{Energía cinética}

Es la energía relacionada con el movimiento. Depende de la velocidad $v$ y de la masa $m$. Su fórmula es:

$$E_C=\dfrac{1}{2}\cdot m \cdot v^2$$


\subsection*{Energía potencial gravitatoria}

Es la energía relacionada con la gravedad. Cuanto más altura tenga el objeto, más será su energía potencial gravitatoria. Su fórmula es:

$$
E_P = m\cdot g \cdot h
$$


\subsection*{Energía potencial elástica}

Es la energía relacionada los resortes o la elasticidad de un objeto. Cuanto más estirado o comprimido esté el objeto, mayor será su energía almacenada. Su fórmula es (siendo $l$ la distancia frente a la longitud natural del resorte):

$$E_E =\dfrac{1}{2} \cdot k \cdot l^2$$


\subsection*{Energía mecánica}

Es la suma entre la energía cinética y las energías potenciales que tiene un cuerpo. Se puede pensar como la energía ``total'' del cuerpo. Su fórmula es (sin considerar la potencial eléctrica y magnética):

$$E_M = E_C + E_G + E_E$$

Su variación es la suma del trabajo realizado por las fuerzas no conservativas:

$$\Delta E_M = \sum W_{F_{\text{NC}}}$$


