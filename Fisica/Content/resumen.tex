\begin{description}
    \item[Mini-glosario] (Las unidades dadas son las más usadas, se pueden usar otras del mismo tipo)\hfill
    
    cte: Constante. Puede ser cualquier número, pero NO varía con el tiempo.
    
    MRU (Movimiento rectilíneo uniforme): un objeto se desplaza en linea recta sin aceleración (velocidad constante).
    
    MRUV (Movimiento rectilíneo uniforme variado): un objeto se desplaza en linea recta con aceleración constante (la velocidad varía linealmente, la posición de manera cuadrática).
    
    $F$: fuerza, sus unidades son Newton (N). 1 N = $1\text{kg}\cdot \frac{\text{m}}{\text{s}^2}$
    
    $m$: masa, sus unidades son kilogramos (kg).
    
    $x(t)$: posición en función del tiempo, sus unidades son metros (m).
    
    $v$: velocidad, sus unidades son metros dividido segundos (m/s).
    
    $v(t)$: velocidad en función del tiempo.
    
    $a$: aceleración , sus unidades son metros divido segundos al cuadrado (m/s$^2$).
    
    $t$: tiempo. Se suele empezar en 0.
    
    $x_0$: posición inicial con respecto a un punto cuando $t=0$. Es constante.
    
    $v_0$: velocidad inicial, cuando $t=0$. Es constante.
    
    \item[Segunda ley de Newton:]\hfill
    
    $$F=m\cdot a$$
    
    \hrule
    
    \item[MRU:]
    
    $$a=0$$
    $$v=\text{cte}$$
    $$x(t)= x_0 + v \cdot t$$
    %%$$E(t)= E_0 + v \cdot (t_f - t_0)$$
    
    \hrule
    
    \item[MRUV:]
    
    $$a=\text{cte}$$
    $$v(t)= v_0 + a\cdot t $$
    $$x(t)=x_0 + v_0\cdot t + \frac{1}{2}\cdot a \cdot t^2$$
    
    \hrule
    
    \item[Trabajo]\hfill
    
    Suele representarse como $W$ (de work) o con $T$. También suele aparecer como $L$. Sus unidades son Joules $J=N\cdot m$. El trabajo que realiza un cuerpo al desplazarse mientras se le aplica una fuerza es igual a la fuerza aplicada multiplicado por la distancia ($d$) que recorrió bajo la acción de esa fuerza:
    
    $$W=F\cdot d$$
    
    En caso que la fuerza sea aplicado en la dirección del desplazamiento, el trabajo es positivo, sino es negativo.
    
    \item[Energía]\hfill
    
    La energía de un cuerpo es la capacidad de realizar trabajo que tiene. Se ven 3 tipos de energía en general: potencial gravitatoria, cinética y mecánica. Al igual que el trabajo, se mide en Joules $J$.
    
    \begin{description}
        \item[Energía cinética]\hfill
        
        Es la energía relacionada con el movimiento. Depende de la velocidad $v$ y de la masa $m$. Su fórmula es:
        
        $$E_C=\dfrac{1}{2}\cdot m \cdot v^2$$
        
        \item[Energía potencial gravitatoria]\hfill
        
        Es la energía relacionada con la gravedad. Cuanto más altura tenga el objeto, más será su energía potencial gravitatoria. Su fórmula es:

        $$
        E_P = m\cdot g \cdot h
        $$

        \item[Energía potencial elástica]\hfill
        
        Es la energía relacionada los resortes o la elasticidad de un objeto. Cuanto más estirado o comprimido esté el objeto, mayor será su energía almacenada. Su fórmula es (siendo $l$ la distancia frente a la longitud natural del resorte):
        
        $$E_E =\dfrac{1}{2} \cdot k \cdot l^2$$
        
        \item[Energía mecánica]\hfill
        
        Es la suma entre la energía cinética y las energías potenciales que tiene un cuerpo. Se puede pensar como la energía ''total´´ del cuerpo. Su fórmula es:
        
        $$E_M = E_C + E_G + E_E$$
        
    \end{description}
    \hrule 
    \item[MCU:]\hfil

    Movimiento Circular Uniforme, es un movimiento en el cual un objeto describe una trayectoria circular, en el cual el tiempo que tarda en dar una vuelta es constante (no hay aceleración tangencial).

    Se trabajará con las siguientes magnitudes: el período $T$, es cuánto se tarda en dar una vuelta. La frecuencia $f$ es cuántas vueltas se dan en un determinado tiempo. la velocidad angular $\omega$ es cuántos grados o radianes se recorren por unidad de tiempo. La velocidad tangencial $v$ es cuántos metros se recorren por unidad de tiempo (es análoga a la velocidad de MRU o MRUV). La aceleración centrípeta $a_c$ es la aceleración que se encarga de modificar la dirección de la velocidad, pero no su módulo.
    
    $$ T = \dfrac{1}{f}
    \hspace{1cm}
    v = \omega\cdot r
    \hspace{1cm}
    \omega = 2\pi \cdot f
    \hspace{1cm}
    a_c = \dfrac{v^2}{r}=\omega^2\cdot r
    \hspace{1cm}
    \theta (t) = \theta _0 + \omega \cdot t
    $$
    \hrule

    \item[MCUV:]\hfil

    Movimiento circular uniformemente variado. Además de una aceleración centrípeta, hay una aceleración tangencial constante.

    $$\theta(t) = \theta_0 + \omega_0 \cdot t + \dfrac{1}{2} \cdot \alpha \cdot t^2$$

    \skipline
    \hrule

    \item[Cantidad de movimiento]\hfill

    La cantidad de movimiento está definido como $p = m\cdot v$. El impulso es igual a la cantida dde movimiento y está definido como $I=F\cdot \Delta t$. Es decir, $p=I=\int F dt$.
    
    \hrule
    
    \item[Análisis de gráficos]\hfill
    
    Teniendo la gráfica de $v(t)$ (Velocidad en función del tiempo), para averiguar la aceleración en un tramo elegís un punto inicial y uno final. Cada punto tiene una componente $X$ y una componente $Y$ (El subíndice $_f$ significa final, el subíndice $_i$ significa inicial). La aceleración la calculás como:
    
    \[a=\dfrac{Y_f - Y_i}{X_f - X_i}\]
    
    Teniendo el mismo gráfico de $v(t)$, la distancia recorrida en cada tramo es el área del tramo (Los cuadrados, triángulos y cuadriláteros que vimos)
    
    \subsubsection*{Geometría:}
    
    El área de un rectángulo es base por altura. El área de un triángulo es base por altura dividido 2. El área de un cuadrilátero la calculás partiendo el cuadrilátero en un cuadrado y un triángulo y sumando el área de ambos.
    \vspace{0.5cm}

    \hrule

    \item[Mini repaso de matemática:]
    
    $$(A\cdot B)^2 = A^2 \cdot B^2$$
    $$\left(\frac{A}{B}\right)^2 = \frac{A^2}{B^2}$$
    $$\sqrt{(A\cdot B)} = \sqrt{A} \cdot \sqrt{B}$$
    $$\sqrt{\left(\frac{A}{B}\right)} = \frac{\sqrt{A}}{\sqrt{B}}$$

    \hrule

    \item[Fuerzas paralelas y colineales]\hfil

    Regla de Stevin: $F_1 \cdot d_1 = F_2 \cdot d_2$. La distancia entre ambas fuerzas es $d$, y $d_n$ es la distancia hasta la posición de la resultante. Las distancias y las fuerzas se ponen con signo positivo, siempre. El valor de la fuerza resultante es la suma de ambas. Si ambas tienen mismo sentido, la resultante estará entre ambas; si tienen sentido opuesto la resultante estará por fuera de ambas.

\skipline
\hrule
\skipline

\item[Campo gravitatorio]\hfil
$$F = \dfrac{G\cdot m_1 \cdot m_2}{d^2}$$

La constante de gravitación universal es $G = 6,67\times 10^{-11}\;\dfrac{\text{N}\cdot \text{m}^2}{\text{kg}^2}$

\item[Campo eléctrico]\hfil

La fórmula de campo eléctrica generado por una carga puntual $q_1$ es:

$$E = \dfrac{k \cdot q_1}{d^2}$$

La fuerza ejercida sobre una carga $q_2$ será:

$$F = \dfrac{k\cdot q_1 \cdot q_2}{d^2} = E \cdot q_2$$ 

La constante de Coulomb $k = 9\times 10^{9}\;\dfrac{\text{N}\cdot \text{m}^2}{\text{C}^2}$

\end{description}

\skipline
\hrule
\skipline

\subsubsection*{Unidades}

\vspace{-3em}
\begin{align*}
t:\;& \text{s}\\
x:\;& \text{m}\\
v:\;& \dfrac{\text{m}}{\text{s}}\\
a:\;& \dfrac{\text{m}}{\text{s}^2}\\
m:\;& \text{kg}\\
F:\;&  \text{N} = \text{kg}\cdot \dfrac{\text{m}}{\text{s}^2}\\
E:\;&  \text{J} = \text{N}\cdot\text{m} = \text{kg}\cdot \dfrac{\text{m}}{\text{s}^2} \cdot \text{m} = \text{kg}\cdot \dfrac{\text{m}^2}{\text{s}^2}\\
P:\;& \text{W} = \dfrac{\text{J}}{s} = \text{kg}\cdot \dfrac{\text{m}^2}{\text{s}^2} \cdot \dfrac{1}{\text{s}} = \text{kg}\cdot \dfrac{\text{m}^2}{\text{s}^3}\\
P:\;&\text{Pa} = \dfrac{\text{N}}{\text{m}^2} = \dfrac{\text{kg}\cdot \text{m}}{\text{s}^2 \cdot \text{m}^2} = \dfrac{\text{kg}}{\text{s}^2 \cdot \text{m}}
\end{align*}

Kilogramo fuerza y Newton:

\begin{align*}
    \text{N} &= \text{kg} \cdot \dfrac{\text{m}}{\text{s}^2}\\
    \text{kgf} &= 9,8 \;\text{N} = 9,8\;\text{kg} \cdot \dfrac{\text{m}}{\text{s}^2} \\
\end{align*}

\newpage

\begin{multicols}{3}
Área:

milímetro cúbico (mm$^2$)

centímetro cúbico (cm$^2$)

decímetro cúbico (dm$^2$)

metro cúbico (m$^2$)

decámetro cúbico (dam$^2$)

hectómetro cúbico (hm$^2$)

kilómetro cúbico (km$^2$)

\vspace{1cm}
Volumen:

milímetro cúbico (mm$^3$)

centímetro cúbico (cm$^3$)

decímetro cúbico (dm$^3$)

metro cúbico (m$^3$)

decámetro cúbico (dam$^3$)

hectómetro cúbico (hm$^3$)

kilómetro cúbico (km$^3$)

\vspace{1cm}
Masa:

miligramo (mg)

centigramo (cg)

decagramo (dg)

gramo (g)

decagramo (dag)

hectogramo (hg)

kilogramo (kg)
\end{multicols}

\subsection*{Engranajes}

Si están en un mismo eje, la frecuencia angular $\omega$ es constante. Si están en contacto o enlazados por una cadena, las velocidades tangenciales $v_t$ son iguales.