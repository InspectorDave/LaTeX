\section{Calorimetría}

$Q$ es calor entregado/quitado al sistema. $m$ masa. $c$ es la capacidad calorífica específica o calor específico, depende del material. $\Delta T$ es la variación de temperatura.

$$ Q = m \cdot C \cdot \Delta T$$

Para realizar un cambio de fase se requiere calor. Siendo $L_F$ y $L_E$ el calor latente de fusión y el de ebullición respectivamente:

$$Q = m \cdot L$$

Para resolver un ejercicio de averiguar temperatura final teniendo todos los datos, plantear que $Q_1=-Q_2$ (calores ganados de cada compuesto), en $\Delta T$ poner como incógnita a $T_f$ y se termina teniendo dos ecuaciones con dos incógnitas. Sumar esas ecuaciones y queda algo fácil igual a 0 (los $Q$ desaparecen).

\skipline
Calores específicos y latentes del agua:

\vspace{0.5\baselineskip}
\hfil
$C_s=0\,5 \dfrac{\text{kcal}}{\text{kg} \cdot \text{ºC}}$\hfil
$C_l=1 \dfrac{\text{kcal}}{\text{kg} \cdot \text{ºC}}$\hfil
$C_g=0,48 \dfrac{\text{kcal}}{\text{kg} \cdot \text{ºC}}$\hfil
$L_f = 80 \dfrac{\text{kcal}}{\text{kg}}$\hfil
$L_e = 540 \dfrac{\text{kcal}}{\text{kg}}$\hfil


\subsection*{Pasaje de unidades en temperaturas}

\begin{multicols}{2}
$^\circ$C y $^\circ$F:
\vspace{-\baselineskip}
\begin{align*}
T_{^\circ\text{F}}&= T_{\text{$^\circ$C}} \cdot 1,8 + 32\\
T_{\text{$^\circ$C}} &= \dfrac{T_{^\circ\text{F}}-32}{1,8}
\end{align*}

K y $^\circ$C:
\vspace{-\baselineskip}
\begin{align*}
T_{\text{K}}&=T_{\text{$^\circ$C}} + 273\\
T_{\text{$^\circ$C}} &=  T_{\text{K}} - 273
\end{align*}
\end{multicols}

\subsection*{Dilatación térmica}

En general dan solo como dato el coeficiente de dilatación lineal $\lambda$. Para obtener el superficial multiplicarlo por dos y para el volumétrico por tres.

La ecuación que relaciona variación de longitud con temperatura es:
$$\Delta L = m \cdot \lambda \cdot \Delta T$$