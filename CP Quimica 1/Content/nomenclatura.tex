\section{Nomenclatura}

\subsection*{\underline{Compuestos binarios}}

\begin{description}

\item[Óxidos básicos (M+O):]\hfil

Se nombran ``Óxido de metal''. Por ejemplo, óxido de sodio (Na$_2$O), óxido de magnesio (MgO).

En caso de haber más de un estado de oxidación existen tres nomenclaturas:

\underline{Tradicional:}

Se nombran ``óxido metal+ oso/ico''. ``oso'' para el menor e ``ico'' para el mayor. Por ejemplo óxido cobaltoso (CoO) y óxido cobáltico (Co$_2$O$_3$).

\underline{Stock:}

Se nombran ``óxido de metal'' y entre números romanos el estado de oxidación del metal. Por ejemplo óxido de cobalto (II) y óxido de cobalto (III).

\underline{Sistemática:}

Se nombran ``óxido de metal'' agregando los prefijos numerales griegos: mono, di, tri, tetra, penta, hexa, hepta, octa, nona y deca a los elementos dependiendo de cuántos átomos haya. El prefijo mono se agrega únicamente al óxido, no al metal. Por ejemplo monóxido de cobalto, trióxido de dicobalto. Para los prefijos que terminan en ``a'' (tetra, penta...), se omite la ``a''.

\skipline
\underline{Obtención:}

Para obtenerlos se hace reaccionar el metal con gas oxígeno, por ejemplo:

\hfil6 Co + 3 O$_2$ $\rightarrow$ 2 Co$_2$O$_3$\hfil





\item[Óxidos ácidos (NM+O):]\hfil

Se nombran ``Óxido de no metal''. En la mayoría de los casos hay más de una estado de oxidación posible del no metal, así que hay que usar una de las tres nomenclaturas:

\underline{Tradicional:}

Se nombran ``óxido hipo/per + no-metal + oso/ico''. En caso de haber 4 posibilidades se usa ``hipo-oso'' para el menor, ``oso'' para el siguiente, ``ico'' para el siguiente y ``per-ico'' para el mayor. En caso de ser 2 posibilidades se usa solo ``-oso'' e ``-ico''. Por ejemplo óxido hipocloroso (Cl$_2$O), óxido cloroso (Cl$_2$O$_3$), óxido clórico (Cl$_2$O$_5$) y óxido perclórico (Cl$_2$O$_7$).

\underline{Stock:}

Se nombran ``óxido de no-metal'' y entre números romanos el estado de oxidación del metal. Por ejemplo óxido de cloro (I), óxido de cloro (III), óxido de cloro (V) y óxido de cloro (VII).

\underline{Sistemática:}

Se nombran ``óxido de metal'' agregando los prefijos numéricos. El prefijo mono se agrega únicamente al óxido, no al no-metal. Por ejemplo monóxido de cloro (Cl$_2$O), monóxido de tricloro (Cl$_2$O$_3$), monóxido de pentacloro (Cl$_2$O$_5$), monóxido de heptacloro (Cl$_2$O$_7$).
\skipline




\newpage
\item[Hidruros metálicos (M+H):]\hfil

El hidrógeno tendrá número de oxidación -1. Se nombran ``hidruro de metal''. Por ejemplo hidruro de sodio (NaH). En caso de haber más de un número de oxidación posible, se puede usar la tradicional o la de Stock:

\underline{Tradicional:}

Hidruro ferroso (FeH$_2$), hidruro férrico (FeH$_3$).

\underline{Stock:}

Hidruro de hierro (II), hidruro de hierro (III).

\underline{Sistemática:}

Se le agregan los prefijos numéricos solo al hidrógeno, siendo siempre habrá un solo metal. Dihidruro de hierro, trihidruro de hierro, monohidruro de cobre, dihidruro de cobre.





\item[Hidruros no metálicos (H+NM):]\hfil

El hidrógeno tendrá número de oxidación +1 y el no metal su único número de oxidación negativo. Se nombran ``no-metal+uro de hidrógeno''. Como existe una única combinación posible entre los elementos, hay una única nomenclatura. Cloruro de hidrógeno HCl, sulfuro de hidrógeno H$_2$S.

\underline{Hidrácidos:}

En caso que el hidruro no metálico sea del grupo 16 o 17 y esté disuelto en agua, se dice que es un hidrácido y se lo nombra ``ácido no-metal+hidríco''. Por ejemplo: ácido clorhídrico (HCl), ácido sulfhídrico (H$_2$S), ácido bromhídrico (HBr).

\underline{Casos especiales:}

Los hidruros no metálicos del grupo 14 y 15 se pueden llamar de manera particular: metano y amoníaco para los del CH$_4$ y NH$_3$, y para el resto se agrega el sufijo ``ano'': silano SiH$_4$, fosfano PH$_3$, arsano AsH$_3$.


Notar que en estos el H se escribe después del elemento, no antes.



\item[Sales binarias (M+NM):]\hfil

El no-metal tendrá su único número de oxidación negativo. Se nombran ``No metal+uro de metal''. Por ejemplo, cloruro de sodio (NaCl), fluoruro de potasio (KF).

En caso de haber más de un estado de oxidación del metal existen tres nomenclaturas:

\underline{Tradicional:}

Se nombran ``no metal+uro de metal oso/ico''. ``oso'' para el menor e ``ico'' para el mayor estado de oxidación del metal. Por ejemplo cloruro cobaltoso (CoO) y cloruro cobáltico (Co$_2$O$_3$).

\underline{Stock:}

Se nombran ``no metal+uro de metal'' y entre números romanos el estado de oxidación del metal. Por ejemplo cloruro de cobalto (II) y cloruro de cobalto (III).

\underline{Sistemática:}

Se nombran ``no metal+uro de metal'' agregando los prefijos numéricos. El prefijo mono se agrega únicamente al primer elemento nombrado, no al metal. Por ejemplo tricloruro de fósforo (PCl$_3$), dinitruro de trihierro (Fe$_3$N$_2$), monosulfuro de disodio (Na$_2$S), dicloruro de hierro (FeCl$_2$).
\end{description}





\newpage
\subsection*{\underline{Compuestos ternarios}}

\begin{description}

\item[Hidróxidos \Big(M+(OH)\Big):]\hfill

    Son un metal con hidroxilo. Habrá un solo átomo de metal y lo que se puede modificar es la cantidad de grupos hidroxilos, no la cantidad de O o H. Se nombran ``hidróxido de metal''. Al escribir la fórmula se ponen paréntesis solo si hay más de un hidroxilo. Por ejemplo hidróxido de sodio NaOH, hidróxido de magnesio Mg(OH)$_2$. En caso de haber más de un número de oxidación del metal:

    \underline{Tradicional:}
    
    Se nombran ``hidróxido de metal oso/ico''. ``oso'' para el menor e ``ico'' para el mayor estado de oxidación del metal. Por ejemplo hidróxido cobaltoso $\Big($Co(OH)$_2\Big)$ o hidróxido cobáltico $\Big(\text{Co}(\text{OH})_3\big)$.
    
    \underline{Stock:}
    
    Se nombran ``hidróxido de metal'' y entre paréntesis con números romanos el estado de oxidación del metal. Por ejemplo hidróxido de cobalto (II) y hidróxido de cobalto (III).
    
    \underline{Sistemática:}
    
    Se nombran ``hidróxido de metal'' agregando los prefijos numéricos griegos al hidróxido. Por ejemplo monohidróxido de oro (AuOH) o trihidróxido de oro $\Big( \text{Au(OH)}_3 \Big)$.





\item[Oxoácidos (H+NM+O):]\hfill
    
    Son un no metal con oxígeno e hidrógeno. En caso de ser un no metal de grupo par, habrá dos H, si es de impar habrá un H (excepto en el fósforo que tiene 3). Como hay más de un número de oxidación posible del no metal hay que usar alguna nomenclatura, la más común es la tradicional. La de Stock y la Sistemática están definidas distinto en varios lugares y no son cómodas.
    
    \underline{Tradicional:}
    
    Se nombran ``ácido hipo/per + no-metal + oso/ico''. Por ejemplo ácido hipocloroso HClO, ácido cloroso HClO$_2$, ácido clórico HClO$_3$, ácido perclórico HClO$_4$; ácido sulfúrico H$_2$SO$_4$, ácido sulfuroso H$_2$SO$_3$; ácido fosfórico H$_3$PO$_4$, ácido fosforoso H$_3$PO$_3$.    



\item[Oxosales (M+NM+O):]\hfill

Son sales que contienen oxígeno. Como el no metal tiene más de un estado de oxidación hay que usar alguna de las nomenclaturas vistas.

    \underline{Tradicional:}
    
    Se nombran ``hipo/per+no metal+ato/ito + metal+oso/ico''. Para el no metal, dependiendo del número de oxidación de menor a mayor es ``hipo-ito'', ``ito'', ``ato'', ''per-ato''. Para el metal se pone ``oso'' o ''ico'' si es que hay más de un estado de oxidación. En caso de ser necesario, se pone el compuesto no metal con oxígeno entre paréntesis. Por ejemplo nitrato de potasio KNO$_3$, nitrito de potasio KNO$_2$; hipoclorito ferroso Fe(ClO)$_2$, hipoclorito férrico Fe(ClO)$_3$, clorito ferroso Fe(ClO$_2$)$_2$, clorito férrico Fe(ClO$_2$)$_3$, clorato ferroso Fe(ClO$_3$)$_2$, clorato férrico Fe(ClO$_3$)$_3$, hipoclorato ferroso Fe(ClO$_4$)$_2$, hipoclorato férrico Fe(ClO$_4$)$_3$; sulfato cobaltoso CoSO$_4$, sulfato cobáltico Co$_2$(SO$_4$)$_3$.
    
    \underline{Stock:}
    
    Se nombran ``hipo/per+no-metal+ito/ato de metal (?)'', donde entre paréntesis va el número de oxidación del metal si tiene más de uno. Algunos autores dicen que el no metal se nombre solo con el sufijo -ato y luego se agrega entre paréntesis su número de oxidación.
    
    \underline{Sistemática:}
    
    La sistemática no se suele usar para oxosales.

\end{description}

