\section{Gases ideales}

Cuando se cambia alguna magnitud del gas (presión, volumen o temperatura) pero la cantidad de gas permanece constante, utilizar la siguiente ecuación (poniendo siempre en todos lados las mismas unidades y la temperatura siempre en Kelvin):

$$ \dfrac{P_1 \cdot V_1}{T_1} = \dfrac{P_2 \cdot V_2}{T_2}$$

En caso de querer averiguar las magnitudes de una determinada cantidad de gas, utilizar la siguiente ecuación (poniendo todo en atm, litros y Kelvin, además R=0,082):

$$ P \cdot V = n \cdot R \cdot T$$

\underline{Algunas aclaraciones:}

\skipline
CNPT: 0$^\circ$ K, 1 atm. Un proceso isotérmico no cambia la temperatura, isobárico no cambia la presión, isocórico no cambia el volumen.
\subsubsection*{Ley de Henry (Gases disueltos en líquido)}

Molaridad es igual a constante (depende del gas, del líquido y de la temperatura) multiplicado por la presión:

$$M = k \cdot P$$

\section{Hidroestática}

\subsubsection*{Presión}

$$P = \dfrac{F}{S}$$

\subsubsection*{Variación de la presión con la profundidad} 

Recordar que la densidad del agua pura es $\delta=1.000 \dfrac{kg}{m^3}$ y la presión atmosférica normal es 1 atm$=101.300$ Pa.

$$P = P_0 + \delta \cdot g \cdot h$$

\subsubsection*{Principio de Arquímides}

``La fuerza flotación de un cuerpo sumergido es igual al peso del líquido desplazado''.

$$F = \delta \cdot V \cdot g$$
