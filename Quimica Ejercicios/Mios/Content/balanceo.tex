\section{Balanceo de ecuaciones}

\begin{enumerate}
\item Balancear las siguientes ecuaciones:
\begin{enumerate}
    \item \ce{H2 + O_2 \rightarrow H_2O}

    \item \ce{CH_4 + O_2 \rightarrow H_2O + CO_2}

    \item \ce{N_2 + H_2 \rightarrow NH_3}

    \item \ce{Cr + O_2 \rightarrow Cr_2O_3}

    \item \ce{KClO_3 \rightarrow KCl + O_2}

    \item \ce{BaCl_2 + Na_2SO_4 \rightarrow NaCl + BaSO_4}

    \item \ce{MgS + AlCl_3 \rightarrow MgCl_2 + Al_2S_3}

    \item \ce{Al + H_2SO_4 \rightarrow Al_2(SO_4)_3 + H_2}

    \item \ce{Cu + HNO_3 \rightarrow Cu(NO_3)_2 + NO + H_2O}

    \item \ce{MnO_2 + HCl \rightarrow MnCl_2 +H_2O +Cl_2}

    \item \ce{H_3PO_4 + NO \rightarrow P_2O_3 + HNO_3 +H_2O} %2 2 1 2 2

    \item \ce{MnO_3 + KOH + AuCl_3 \rightarrow AuCl + KCl + KMnO_4 + H_2O} %2 4 1 1 2 2 2
\end{enumerate}
\end{enumerate}

\subsection{Método algebraico}

\underline{Ejemplo:}

$$\text{Fe} + \text{HCl} \longrightarrow \text{FeCl}_3 + \text{H}_2$$

\begin{minipage}[l]{0.2\textwidth}
\begin{align*}
\text{Fe}&: A = C\\
\text{H}&: B = 2D\\
\text{Cl}&: B = 3C\\
\;
\end{align*}
\end{minipage}
$\Longrightarrow$
\hfil
$B=2D=3C=3A$
\hfil
$\Longrightarrow$
\begin{minipage}[l]{0.1\textwidth}
\begin{align*}
A&= 2\\
B&= 6\\
C&= 2\\
D&= 3
\end{align*}
\end{minipage}
$\Longrightarrow$
\hfil 
\fbox{$2\text{Fe} + 6\text{HCl} \longrightarrow 2\text{FeCl}_3 + 3\text{H}_2$}
\hfil


\subsection{Redox}

\begin{enumerate}

\item Balancear las siguientes semireacciones:
\begin{enumerate}
    \item $\ce{ NO_3^- -> N_2 }$ (Medio ácido)
    \item $\ce{ Fe^{2+} -> Fe^{3+} }$
    \item $\ce{ SO_2^{\;2-} -> SO_3^{\;2-} }$ (medio básico)
    \item $\ce{ SO_2^{\;2-} -> SO_3^{\;2-} }$ (medio ácido)
    \item $\ce{ N_2 -> NO_2^- }$ (medio básico)
\end{enumerate}


\item Balancear por redox:
$$3\text{As}_2\text{O}_3 + 4\text{HNO}_3 + 7\text{H}_2\text{O} \rightarrow
6\text{H}_3 \text{AsO}_4 + 4\text{NO}$$

Se ve que As se oxida y que N se reduce.

\begin{multicols}{2}

\underline{Semirreacción de oxidación:}
$$\text{As}_2\text{O}_3 + 5\text{H}_2\text{O}\longrightarrow
2(\text{AsO}_4)^{3-} + 10\text{H}^+ + 4e^-$$

\underline{Semirreacción de reducción:}
$$(\text{NO}_3)^{-} + 4 \text{H}^+ \longrightarrow
\text{NO} + 2\text{H}_2\text{O} - 3e^-$$
\end{multicols}

Ahora multiplico la semirreacción de oxidación por 3 y la de reducción por 4, de manera que al sumarlas se cancelen los $e^-$:

\begin{multicols}{2}

\underline{Semirreacción de oxidación:}
$$\big(\text{As}_2\text{O}_3 + 5\text{H}_2\text{O}\longrightarrow
2(\text{AsO}_4)^{3-} + 10\text{H}^+ + 4e^-\big) \cdot 3$$

\underline{Semirreacción de reducción:}
$$\big((\text{NO}_3)^{-} + 4 \text{H}^+ \longrightarrow
\text{NO} + 2\text{H}_2\text{O} - 3e^-\big) \cdot 4$$
\end{multicols}

\noindent
Las sumo y escribo la respuesta:

$$\fbox{\ce{3As_2O_3 + 4HNO_3 + 7H_2O -> 6H_3AsO_4 + 4NO}}$$


\item Balancear por redox:
$$\ce{MnO_2 + HCl \rightarrow MnCl_2 + H_2O + Cl_2}$$

\underline{Escribo los números de oxidación}
$$\ce{Mn^{4+}O_2^{2-} + H^{+}Cl^{-} \rightarrow Mn^{2+}Cl_2^{-} + H_2^+O^{2-} + Cl_2^{0}}$$

\begin{multicols}{2}
    \underline{Semirreacción de oxidación:}
    $$\ce{2Cl^- ->
    Cl_2 + 2e^-}$$
    
    \underline{Semirreacción de reducción:}
    $$\ce{MnO_2 + 4 H^+ + 2e^- ->
    Mn^{2+} + 2H_2O}$$
\end{multicols}

Como las cantidades de \ce{e^-} ya son iguales, las sumo. No olvidar considerar bien los Cl al momento de terminar de balencear.

$$\ce{2 Cl^- + MnO_2 + 4H^+ + 2e^- -> 2e^- + Mn^{2+} + 2H_2O + Cl_2}$$

Finalmente cancelo los electrones y a partir de los iones escribo los coeficientes de las sustancias:

$$\fbox{\ce{MnO_2 + 4HCl -> MnCl_2 + 2H_2O + Cl_2}}$$


\item Balancear por Redox:
$$\text{K}_2\text{Cr}_2\text{O}_7 + \text{KI} + \text{H}_2\text{SO}_4 \longrightarrow \text{K}_2\text{SO}_4 + \text{I}_2 + \text{Cr}_2(\text{SO}_4)_3 + \text{H}_2\text{O}$$


\item Balancear la ecuación que tiene los siguientes reactivos y productos:

\underline{Reactivos}:

permanganato de potasio (KMnO$_4$), hidróxido de potasio (KOH), yoduro de potasio (KI).

\skipline
\underline{Productos}:

Yodato de potasio (KIO$_3$), manganato (VI) de potasio (I) (K$_2$MnO$_4$)

\skipline
{\large
\hfil
K$^{+1}$Mn$^{+7}$O$_4^{-2}$ \hfil+\hfil
K$^{+1}$O$^{-2}$H$^{+1}$ \hfil+\hfil
K$^{+1}$I$^{-1}$ \hfil$\longrightarrow$\hfil
K$^{+1}$I$^{+5}$O$_3^{-2}$ \hfil+\hfil
K$_2^{+1}$Mn$^{+6}$O$_4^{-2}$
\hfil
}


\item Balancear por Redox (7 del CNBA):
$$\ce{FeSO_4 + HNO_3 + H_2SO_4 -> Fe_2(SO4)3 + NO + H2O}$$

\begin{multicols}{2}
\underline{Semireacción de reducción:}
 $$\left(\ce{NO_3^- + 4H^+ + 3e^- -> NO + 2H_2O}\right)\cdot 2$$

\underline{Semireacción de oxidación:}
$$\left(\ce{ 2Fe^{2+} -> Fe_2^{3+} + 2e^-}  \right)\cdot 3$$
\end{multicols}

Las sumo:
$$
\ce{6Fe^{2+} + 2NO_3^- + 8H^+ -> 3Fe_2^{3+} + 2NO + 4H_2O}
$$

Pongo los coeficientes en la ecuación original, tener en cuenta que los H$^+$ se distribuyen entre los ácidos:
$$\fbox{\ce{6FeSO_4 + 2HNO_3 + 3H_2SO_4 -> 3Fe_2(SO4)3 + 2NO + 4H2O}}$$


\item Balancear por Redox (8 del CNBA):
$$\ce{
As2O3 + HNO3 + H2O -> H3AsO4 + NO
}$$

Escribo los números de oxidación:
$$\ce{
As2^{+3}O^{-2}3 + H^{+1}N^{+5}O^{-2}3 + H^{+1}2O^{-2} -> H^{+1}3As^{+5}O4^{-2} + N^{+2}O^{-2}
}$$

\hfil Se oxida: As\hfil Agente reductor: \ce{As2O3}\hfil

\hfil Se reduce: N\hfil Agente oxidante: \ce{HNO3}\hfil

Es Medio ácido.

\begin{multicols}{2}
Semirreacción de reducción:
$$\left(\ce{
As2O3 + 5H2O -> 2AsO4^{3-} + 10H^+ +4e^-
}\right)\cdot 3$$

Semirreacción de oxidación:
$$\left(\ce{
NO_3^{-} + 4H^+ +3e^- -> NO + 2H2O
}\right)\cdot 4$$
\end{multicols}

Las sumo:
$$\ce{
3As2O3 + 15H2O + 4NO3^- + 16H^+ -> 6AsO_4^{3-} + 30H^+ + 4NO + 8 H2O
}$$
$$\ce{
3As2O3 + 7H2O + 4NO3^- -> 6AsO_4^{3-} + 14H^+ + 4NO
}$$

Pongo los coeficientes:
$$\fbox{\ce{
3As2O3 + 4HNO3 + 7H2O -> 6H3AsO4 + 4NO}}$$


\newpage
\item Balancear por Redox (9 del CNBA):
$$\ce{K2SO3 + KNO3 + H2O ->
K2SO4 + N2O + KOH}$$

Escribo los números de oxidación, identifico cuál se oxida, cuál se reduce, los agentes y el medio:
$$\ce{
K^{+1}2S^{+4}O^{-2}3 + K^{+1}N^{+5}O^{-2}3 + H^{+1}2O^{-2} ->
K^{+1}2S^{+6}O^{-2}4 + N^{+1}2O^{-2} + K^{+1}O^{-2}H^{-1}
}$$

\hfil Se oxida: S\hfil Agente reductor: \ce{K2SO3}\hfil

\hfil Se reduce: N\hfil Agente oxidante: \ce{KNO3}\hfil

Es Medio básico.

\begin{multicols}{2}
Semirreacción de reducción:
$$\ce{
2NO_3^{-} + 5 H2O + 8e^- ->
N2O + 10OH^-
}$$

Semirreacción de oxidación:
$$\left(\ce{
SO3^{2-} + 2OH^- ->
SO4^{2-} + H2O + 2e^-
}\right)\cdot 4$$
\end{multicols}

Las sumo y simplifico:
$$\ce{
4SO3^{2-} + 8OH^- + 2NO_3^- + 5H2O ->
4SO_4^{2-} + 4 H2O + N2O + 10OH^-
}$$
$$\ce{
4SO3^{2-} + 2NO_3^- + H2O ->
4SO_4^{2-} + N2O + 2OH^-
}$$

Finalmente pongo los coeficientes:
$$\fbox{\ce{4K2SO3 + 2KNO3 + H2O ->
4K2SO4 + N2O + 2KOH}}$$


\item
Balancear por Redox (10 del CNBA):
$$\ce{
HNO3 + SnCl2 + HCl ->
SnCl4 + N2O + H2O
}$$

Escribo los números de oxidación, identifico cuál se oxida, cuál se reduce, los agentes y el medio:
$$\ce{
H^{+1}N^{+5}O^{-2}3 + Sn^{+2}Cl^{-1}2 + H^{+1}Cl^{-1} ->
Sn^{+4}Cl^{-1}4 + N^{+2}2O^{-2} + H^{+1}2O^{-2}
}$$

\hfil Se reduce: N\hfil Agente oxidante: \ce{HNO3}\hfil

\hfil Se oxida: Sn\hfil Agente reductor: \ce{SnCl2}\hfil

Es Medio ácido.

\begin{multicols}{2}
Semirreacción de reducción:
$$\ce{
2NO_3^- +10H^+ + 8e^- ->
N2O + 5H2O
}$$

Semirreacción de oxidación:
$$\left(\ce{
Sn^{2+} ->
Sn^{4+} + 2e^-
}\right)\cdot 4$$
\end{multicols}

Las sumo y simplifico:
$$\ce{
2NO_3^- + 10H^+ + 4Sn^{2+} ->
N2O + 5H2O + 4Sn^{4+}
}$$

Finalmente pongo los coeficientes:
$$\fbox{\ce{
2HNO3 + 4SnCl2 + 8HCl ->
4SnCl4 + N2O + 5H2O
}}$$

\end{enumerate}
