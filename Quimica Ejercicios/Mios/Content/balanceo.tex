\section{Balanceo de ecuaciones}

\begin{enumerate}
\item Balancear las siguientes ecuaciones:
\begin{enumerate}
    \item \ce{H2 + O_2 \rightarrow H_2O}

    \item \ce{CH_4 + O_2 \rightarrow H_2O + CO_2}

    \item \ce{N_2 + H_2 \rightarrow NH_3}

    \item \ce{Cr + O_2 \rightarrow Cr_2O_3}

    \item \ce{KClO_3 \rightarrow KCl + O_2}

    \item \ce{BaCl_2 + Na_2SO_4 \rightarrow NaCl + BaSO_4}

    \item \ce{MgS + AlCl_3 \rightarrow MgCl_2 + Al_2S_3}

    \item \ce{Al + H_2SO_4 \rightarrow Al_2(SO_4)_3 + H_2}

    \item \ce{Cu + HNO_3 \rightarrow Cu(NO_3)_2 + NO + H_2O}

    \item \ce{MnO_2 + HCl \rightarrow MnCl_2 +H_2O +Cl_2}

    \item \ce{H_3PO_4 + NO \rightarrow P_2O_3 + HNO_3 +H_2O} %2 2 1 2 2

    \item \ce{MnO_3 + KOH + AuCl_3 \rightarrow AuCl + KCl + KMnO_4 + H_2O} %2 4 1 1 2 2 2
\end{enumerate}
\end{enumerate}

\subsection{Método algebraico}

\underline{Ejemplo:}

$$\text{Fe} + \text{HCl} \longrightarrow \text{FeCl}_3 + \text{H}_2$$

\begin{minipage}[l]{0.2\textwidth}
\begin{align*}
\text{Fe}&: A = C\\
\text{H}&: B = 2D\\
\text{Cl}&: B = 3C\\
\;
\end{align*}
\end{minipage}
$\Longrightarrow$
\hfil
$B=2D=3C=3A$
\hfil
$\Longrightarrow$
\begin{minipage}[l]{0.1\textwidth}
\begin{align*}
A&= 2\\
B&= 6\\
C&= 2\\
D&= 3
\end{align*}
\end{minipage}
$\Longrightarrow$
\hfil 
\fbox{$2\text{Fe} + 6\text{HCl} \longrightarrow 2\text{FeCl}_3 + 3\text{H}_2$}
\hfil


\subsection{Redox}

\begin{enumerate}

\item Balancear las siguientes semireacciones:
\begin{enumerate}
    \item $\ce{ NO_3^- -> N_2 }$ (Medio ácido)
    \item $\ce{ Fe^{2+} -> Fe^{3+} }$
    \item $\ce{ SO_2^{\;2-} -> SO_3^{\;2-} }$ (medio básico)
    \item $\ce{ SO_2^{\;2-} -> SO_3^{\;2-} }$ (medio ácido)
    \item $\ce{ N_2 -> NO_2^- }$ (medio básico)
\end{enumerate}


\item Balancear por redox:
$$3\text{As}_2\text{O}_3 + 4\text{HNO}_3 + 7\text{H}_2\text{O} \rightarrow
6\text{H}_3 \text{AsO}_4 + 4\text{NO}$$

Se ve que As se oxida y que N se reduce.

\begin{multicols}{2}

\underline{Semirreacción de oxidación:}
$$\text{As}_2\text{O}_3 + 5\text{H}_2\text{O}\longrightarrow
2(\text{AsO}_4)^{3-} + 10\text{H}^+ + 4e^-$$

\underline{Semirreacción de reducción:}
$$(\text{NO}_3)^{-} + 4 \text{H}^+ \longrightarrow
\text{NO} + 2\text{H}_2\text{O} - 3e^-$$
\end{multicols}

Ahora multiplico la semirreacción de oxidación por 3 y la de reducción por 4, de manera que al sumarlas se cancelen los $e^-$:

\begin{multicols}{2}

\underline{Semirreacción de oxidación:}
$$\big(\text{As}_2\text{O}_3 + 5\text{H}_2\text{O}\longrightarrow
2(\text{AsO}_4)^{3-} + 10\text{H}^+ + 4e^-\big) \cdot 3$$

\underline{Semirreacción de reducción:}
$$\big((\text{NO}_3)^{-} + 4 \text{H}^+ \longrightarrow
\text{NO} + 2\text{H}_2\text{O} - 3e^-\big) \cdot 4$$
\end{multicols}

\noindent
Las sumo y escribo la respuesta:

$$\fbox{\ce{3As_2O_3 + 4HNO_3 + 7H_2O -> 6H_3AsO_4 + 4NO}}$$

\item Balancear por redox:

$$\ce{MnO_2 + HCl \rightarrow MnCl_2 + H_2O + Cl_2}$$

\underline{Escribo los números de oxidación}

$$\ce{Mn^{4+}O_2^{2-} + H^{+}Cl^{-} \rightarrow Mn^{2+}Cl_2^{-} + H_2^+O^{2-} + Cl_2^{0}}$$

\begin{multicols}{2}
    \underline{Semirreacción de oxidación:}
    $$\ce{2Cl^- ->
    Cl_2 + 2e^-}$$
    
    \underline{Semirreacción de reducción:}
    $$\ce{MnO_2 + 4 H^+ + 2e^- ->
    Mn^{2+} + 2H_2O}$$
\end{multicols}

Como las cantidades de \ce{e^-} ya son iguales, las sumo. No olvidar considerar bien los Cl al momento de terminar de balencear.

$$\ce{2 Cl^- + MnO_2 + 4H^+ + 2e^- -> 2e^- + Mn^{2+} + 2H_2O + Cl_2}$$

Finalmente cancelo los electrones y a partir de los iones escribo los coeficientes de las sustancias:

$$\fbox{\ce{MnO_2 + 4HCl -> MnCl_2 + 2H_2O + Cl_2}}$$


\item Balancear por Redox:
$$\text{K}_2\text{Cr}_2\text{O}_7 + \text{KI} + \text{H}_2\text{SO}_4 \longrightarrow \text{K}_2\text{SO}_4 + \text{I}_2 + \text{Cr}_2(\text{SO}_4)_3 + \text{H}_2\text{O}$$


\item 
Balancear la ecuación que tiene los siguientes reactivos y productos:

\underline{Reactivos}:

permanganato de potasio (KMnO$_4$), hidróxido de potasio (KOH), yoduro de potasio (KI).

\skipline
\underline{Productos}:

Yodato de potasio (KIO$_3$), manganato (VI) de potasio (I) (K$_2$MnO$_4$)

\skipline
{\large
\hfil
K$^{+1}$Mn$^{+7}$O$_4^{-2}$ \hfil+\hfil
K$^{+1}$O$^{-2}$H$^{+1}$ \hfil+\hfil
K$^{+1}$I$^{-1}$ \hfil$\longrightarrow$\hfil
K$^{+1}$I$^{+5}$O$_3^{-2}$ \hfil+\hfil
K$_2^{+1}$Mn$^{+6}$O$_4^{-2}$
\hfil
}
\end{enumerate}
