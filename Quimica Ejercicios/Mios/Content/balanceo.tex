\section{Balanceo de ecuaciones}

\begin{enumerate}
\item Balancear las siguientes ecuaciones:
\begin{enumerate}
    \item \ce{H2 + O_2 \rightarrow H_2O}

    \item \ce{CH_4 + O_2 \rightarrow H_2O + CO_2}

    \item \ce{N_2 + H_2 \rightarrow NH_3}

    \item \ce{Cr + O_2 \rightarrow Cr_2O_3}

    \item \ce{KClO_3 \rightarrow KCl + O_2}

    \item \ce{BaCl_2 + Na_2SO_4 \rightarrow NaCl + BaSO_4}

    \item \ce{MgS + AlCl_3 \rightarrow MgCl_2 + Al_2S_3}

    \item \ce{Al + H_2SO_4 \rightarrow Al_2(SO_4)_3 + H_2}

    \item \ce{Cu + HNO_3 \rightarrow Cu(NO_3)_2 + NO + H_2O}

    \item \ce{MnO_2 + HCl \rightarrow MnCl_2 +H_2O +Cl_2}

    \item \ce{H_3PO_4 + NO \rightarrow P_2O_3 + HNO_3 +H_2O} %2 2 1 2 2

    \item \ce{MnO_3 + KOH + AuCl_3 \rightarrow AuCl + KCl + KMnO_4 + H_2O} %2 4 1 1 2 2 2
\end{enumerate}
\end{enumerate}


\subsection*{Redox}

\begin{enumerate}
\item Balancear las siguientes semireacciones:
\begin{enumerate}
    \item $\ce{ NO_3^- -> N_2 }$ (Medio ácido)
    \item $\ce{ Fe^{2+} -> Fe^{3+} }$
    \item $\ce{ SO_2^{\;2-} -> SO_3^{\;2-} }$ (medio básico)
    \item $\ce{ SO_2^{\;2-} -> SO_3^{\;2-} }$ (medio ácido)
    \item $\ce{ N_2 -> NO_2^- }$ (medio básico)
\end{enumerate}


\item Balancear por Redox:
$$\text{K}_2\text{Cr}_2\text{O}_7 + \text{KI} + \text{H}_2\text{SO}_4 \longrightarrow \text{K}_2\text{SO}_4 + \text{I}_2 + \text{Cr}_2(\text{SO}_4)_3 + \text{H}_2\text{O}$$


\item Balancear por Redox la ecuación que tiene los siguientes reactivos y productos:

\underline{Reactivos}:

permanganato de potasio (KMnO$_4$), hidróxido de potasio (KOH), yoduro de potasio (KI).

\skipline
\underline{Productos}:

Yodato de potasio (KIO$_3$), manganato (VI) de potasio (I) (K$_2$MnO$_4$)
\end{enumerate}


\newpage
\subsubsection*{Ejercicios del CNBA}

\begin{enumerate}
\item Balancear por redox:
$$\ce{MnO_2 + HCl \rightarrow MnCl_2 + H_2O + Cl_2}$$

\underline{Escribo los números de oxidación}
$$\ce{Mn^{+4}O_2^{-2} + H^{+1}Cl^{-1} \rightarrow Mn^{+2}Cl_2^{-1} + H_2^{+1}O^{-2} + Cl_2^{0}}$$

\begin{multicols}{2}
    \underline{Semirreacción de oxidación:}
    $$\ce{2Cl^- ->
    Cl_2 + 2e^-}$$
    
    \underline{Semirreacción de reducción:}
    $$\ce{MnO_2 + 4 H^+ + 2e^- ->
    Mn^{2+} + 2H_2O}$$
\end{multicols}

Como las cantidades de \ce{e^-} ya son iguales, las sumo. No olvidar considerar bien los Cl al momento de terminar de balencear.

$$\ce{2 Cl^- + MnO_2 + 4H^+ + 2e^- -> 2e^- + Mn^{2+} + 2H_2O + Cl_2}$$

Finalmente cancelo los electrones y a partir de los iones escribo los coeficientes de las sustancias:
$$\fbox{\ce{MnO_2 + 4HCl -> MnCl_2 + 2H_2O + Cl_2}}$$


\item Balancear por Redox:
$$\ce{
KI + KIO3 + HCl ->
KCl + I2 + H2O
}$$

Escribo los números de oxidación, identifico cuál se oxida, cuál se reduce, los agentes y el medio: 
$$\ce{
K^{+1}I^{-1} + K^{+1}I^{+5}O^{-2}3 + H^{+1}Cl^{-1} ->
K^{+1}Cl^{-1} + I^{0}2 + H^{+1}2O^{-2}
}$$

\medioreduceoxida{ácido}{I}{KIO3}{I}{KI}

\begin{multicols}{2}
Semirreacción de reducción:

\hfil$2(\text{IO}_3)^{-} + 12 \text{H}^+ \longrightarrow \text{I}_2 + 6\text{H}_2 \text{O} - 10e^-$\hfil

Semirreacción de oxidación:
$$\left(2\text{I}^{-} \longrightarrow \text{I}_2 + 2e^-\right)\cdot 5$$
\end{multicols}

Las sumo y simplifico:

\hfil$\ce{2(IO3)^- + 12H^+ + 10 I^- -> I2 + 6H2O + 5I2}$\hfil

\hfil$\ce{(IO3)^- + 6H^+ + 5 I^- -> 3I2 + 3H2O }$\hfil

Finalmente pongo los coeficientes en la fórmula original y agrego lo que haga falta:
\[\boxed{5\text{KI} + \text{K} \text{I} \text{O}_3 + 6\text{H} \text{Cl} \longrightarrow
6\text{KCl} + 3\text{I}_2 +  3\text{H}_2 \text{O}}\]


\newpage
\item
Balancear por Redox:
$$\ce{
Cu + HNO3 ->
Cu(NO3)2 + NO2 + H2O
}$$

Escribo los números de oxidación, identifico cuál se oxida, cuál se reduce, los agentes y el medio:
$$\ce{
Cu^0 + H^{+1}N^{+5}O^{-2}3 ->
Cu^{+2}(N^{+5}O^{-2}3)2 + N^{+4}O^{-2}2 + H^{+1}2O^{-2}
}$$

\medioreduceoxida{ácido}{N}{HNO3}{Cu}{Cu}

\begin{multicols}{2}
Semirreacción de reducción:
$$\left(\ce{
NO3^{-} + 2H^+ + e^- ->
NO2 + H2O
}\right)\cdot 2$$

Semirreacción de oxidación: 
$$\ce{
Cu ->
Cu^{2+} + 2e^-
}$$
\end{multicols}

Las sumo y simplifico:
$$\ce{
2NO3^{-} + 4H^+ + Cu ->
2NO2 + 2 H2O + Cu^{2+}
}$$

Finalmente pongo los coeficientes (ver que se pone 4 en vez de 2 en \ce{HNO3} porque hay \ce{NO3^}):
$$\fbox{\ce{
Cu + 4HNO3 ->
Cu(NO3)2 + 2NO2 + 2H2O
}}$$


\item
Balancear por Redox:
$$\ce{
KMnO4 + KOH + KI ->
K2MnO4 + KIO3 + H2O
}$$

Escribo los números de oxidación, identifico cuál se oxida, cuál se reduce, los agentes y el medio:
\[\ce{
K^{+1}Mn^{+7}O^{-2}4 + K^{+1}O^{-2}H^{+1} + K^{+1}I^{-1} ->
K^{+1}2Mn^{+6}O^{-2}4 + K^{+1}I^{+5}O^{-2}3 + H^{+1}2O^{-2}
}\]

\medioreduceoxida{básico}{Mn}{KMnO4}{I}{KI}

\begin{multicols}{2}
Semirreacción de reducción:
$$\left(\ce{
MnO4^- +e^- ->
MnO4^{2-}
}\right)\cdot 6$$

Semirreacción de oxidación:
$$\ce{
I^{-} + 6OH^- ->
IO3^- + 3H2O + 6e^-
}$$
\end{multicols}

Las sumo y simplifico:
$$\ce{
6MnO4^- + I^- + 6OH^- ->
6MnO4^{2-} + IO3^- + 3H2O
}$$

Finalmente pongo los coeficientes:
$$\fbox{\ce{
6KMnO4 + 6KOH + KI ->
6K2MnO4 + KIO3 + 3H2O
}}$$


\newpage 
\item
Balancear por Redox:
$$\ce{
KMnO4 + KOH + KAsO2 ->
MnO2 + K3AsO4 + H2O
}$$

Escribo los números de oxidación, identifico cuál se oxida, cuál se reduce, los agentes y el medio:
\[\ce{
K^{+1}Mn^{+7}O^{-2}4 + K^{+1}O^{-2}H^{+1} + K^{+1}As^{+3}O^{-2}2 ->
Mn^{+4}O^{-2}2 + K^{+1}3As^{+5}O^{-2}4 + H^{+1}2O^{-2}
}\]

\medioreduceoxida{básico}{Mn}{KMnO4}{As}{KAsO2}

\begin{multicols}{2}
Semirreacción de reducción:
$$\left(\ce{
MnO4^- + 2H2O + 3e^- ->
MnO2 + 4OH^-
}\right)\cdot 2$$

Semirreacción de oxidación:
$$\left(\ce{
AsO2^- + 4OH^- ->
AsO_4^{3-} + 2H2O + 2e^-
}\right)\cdot 3$$
\end{multicols}

Las sumo y simplifico:
$$\ce{
2MnO4^- + 4 H2O + 3AsO2^- + 12 OH^- ->
2MnO2 + 8OH^- + 3 AsO4^{3-} + 6H2O
}$$

Finalmente pongo los coeficientes:
$$\fbox{\ce{
2KMnO4 + 4KOH + 3KAsO2 ->
2MnO2 + 3K3AsO4 + 2H2O
}}$$


\item
Balancear por Redox:
$$\ce{
KClO3 + CrCl3 + KOH ->
K2CrO4 + H2O + KCl
}$$

Escribo los números de oxidación, identifico cuál se oxida, cuál se reduce, los agentes y el medio:
$$\ce{
K^{+1}Cl^{+5}O^{-2}3 + Cr^{+3}Cl^{-1}3 + K^{+1}O^{-2}H^{+1} ->
K^{+1}2Cr^{+6}O^{-2}4 + H^{+1}2O^{-2} + K^{+1}Cl^{-1}
}$$

\medioreduceoxida{básico}{Cl}{KClO3}{Cr}{CrCl3}

\begin{multicols}{2}
Semirreacción de reducción:
$$\ce{
ClO3^- + 3H2O + 6e^- ->
Cl^- + 6OH^-
}$$

Semirreacción de oxidación:
$$\left( \ce{
Cr^{3+} +8OH^- ->
CrO_4^{2-} + 4H2O + 3e^-
}\right) \cdot 2$$
\end{multicols}

Las sumo y simplifico:

\hfil$\ce{
ClO3^- + 3H2O + 2Cr^{3+} + 16OH^- ->
Cl^- + 6OH^- + 2CrO_4^{2-} + 8H2O
}$\hfil

\hfil$\ce{
ClO3^- + 2Cr^{3+} + 10OH^- ->
Cl^- + 2CrO_4^{2-} + 5H2O
}$\hfil

Finalmente pongo los coeficientes (el del \ce{KCl} se pone al tanteo):
$$\fbox{\ce{
KClO3 + 2CrCl3 + 10KOH ->
2K2CrO4 + 5H2O + 7KCl
}}$$


\newpage
\item Balancear por Redox:
$$\ce{FeSO4 + HNO3 + H2SO4 -> Fe2(SO4)3 + NO + H2O}$$

\medioreduceoxida{ácido}{N}{HNO3}{Fe}{FeSO4}

\begin{multicols}{2}
\underline{Semireacción de reducción:}
 $$\left(\ce{NO_3^- + 4H^+ + 3e^- -> NO + 2H_2O}\right)\cdot 2$$

\underline{Semireacción de oxidación:}
$$\left(\ce{ 2Fe^{2+} -> Fe_2^{3+} + 2e^-}  \right)\cdot 3$$
\end{multicols}

Las sumo:
$$
\ce{6Fe^{2+} + 2NO_3^- + 8H^+ -> 3Fe_2^{3+} + 2NO + 4H_2O}
$$

Pongo los coeficientes en la ecuación original, tener en cuenta que los H$^+$ se distribuyen entre los ácidos:
$$\fbox{\ce{6FeSO_4 + 2HNO_3 + 3H_2SO_4 -> 3Fe_2(SO4)3 + 2NO + 4H2O}}$$


\item Balancear por Redox:
$$\ce{
As2O3 + HNO3 + H2O -> H3AsO4 + NO
}$$

Escribo los números de oxidación:
$$\ce{
As2^{+3}O^{-2}3 + H^{+1}N^{+5}O^{-2}3 + H^{+1}2O^{-2} -> H^{+1}3As^{+5}O4^{-2} + N^{+2}O^{-2}
}$$

\medioreduceoxida{ácido}{As}{As2O3}{N}{HNO3}

\begin{multicols}{2}
Semirreacción de reducción:
$$\left(\ce{
As2O3 + 5H2O -> 2AsO4^{3-} + 10H^+ +4e^-
}\right)\cdot 3$$

Semirreacción de oxidación:
$$\left(\ce{
NO_3^{-} + 4H^+ +3e^- -> NO + 2H2O
}\right)\cdot 4$$
\end{multicols}

Las sumo:
$$\ce{
3As2O3 + 15H2O + 4NO3^- + 16H^+ -> 6AsO_4^{3-} + 30H^+ + 4NO + 8 H2O
}$$
$$\ce{
3As2O3 + 7H2O + 4NO3^- -> 6AsO_4^{3-} + 14H^+ + 4NO
}$$
Pongo los coeficientes:
$$\fbox{\ce{
3As2O3 + 4HNO3 + 7H2O -> 6H3AsO4 + 4NO}}$$


\newpage
\item
Balancear por Redox:
$$\ce{K2SO3 + KNO3 + H2O ->
K2SO4 + N2O + KOH}$$

Escribo los números de oxidación, identifico cuál se oxida, cuál se reduce, los agentes y el medio:
$$\ce{
K^{+1}2S^{+4}O^{-2}3 + K^{+1}N^{+5}O^{-2}3 + H^{+1}2O^{-2} ->
K^{+1}2S^{+6}O^{-2}4 + N^{+1}2O^{-2} + K^{+1}O^{-2}H^{-1}
}$$

\medioreduceoxida{básico}{S}{K2SO3}{N}{KNO3}

\begin{multicols}{2}
Semirreacción de reducción:
$$\ce{
2NO_3^{-} + 5 H2O + 8e^- ->
N2O + 10OH^-
}$$
Semirreacción de oxidación:
$$\left(\ce{
SO3^{2-} + 2OH^- ->
SO4^{2-} + H2O + 2e^-
}\right)\cdot 4$$
\end{multicols}

Las sumo y simplifico:
$$\ce{
4SO3^{2-} + 8OH^- + 2NO_3^- + 5H2O ->
4SO_4^{2-} + 4 H2O + N2O + 10OH^-
}$$
$$\ce{
4SO3^{2-} + 2NO_3^- + H2O ->
4SO_4^{2-} + N2O + 2OH^-
}$$

Finalmente pongo los coeficientes:
$$\fbox{\ce{4K2SO3 + 2KNO3 + H2O ->
4K2SO4 + N2O + 2KOH}}$$

\item
Balancear por Redox:
$$\ce{
HNO3 + SnCl2 + HCl ->
SnCl4 + N2O + H2O
}$$

Escribo los números de oxidación, identifico cuál se oxida, cuál se reduce, los agentes y el medio:
$$\ce{
H^{+1}N^{+5}O^{-2}3 + Sn^{+2}Cl^{-1}2 + H^{+1}Cl^{-1} ->
Sn^{+4}Cl^{-1}4 + N^{+2}2O^{-2} + H^{+1}2O^{-2}
}$$

\medioreduceoxida{ácido}{N}{HNO4}{Sn}{SnCl2}

\begin{multicols}{2}
Semirreacción de reducción:
$$\ce{
2NO_3^- +10H^+ + 8e^- ->
N2O + 5H2O
}$$

Semirreacción de oxidación:
$$\left(\ce{
Sn^{2+} ->
Sn^{4+} + 2e^-
}\right)\cdot 4$$
\end{multicols}

Las sumo y simplifico:
$$\ce{
2NO_3^- + 10H^+ + 4Sn^{2+} ->
N2O + 5H2O + 4Sn^{4+}
}$$

Finalmente pongo los coeficientes:

\hfil\fbox{\ce{
2HNO3 + 4SnCl2 + 8HCl ->
4SnCl4 + N2O + 5H2O
}}\hfil


\item 
Balancear por Redox:
$$\ce{KMnO4 + H2O2 + H2SO4 -> O2 + MnSO4 + K2SO4 + H2O}$$
Escribo los números de oxidación:
$$\ce{
K^{+1}Mn^{+7}O^{-2}4 + H^{+1}2O^{-1}2 + H^{+1}2S^{+6}O^{-2}4 ->
O^{0}2 + Mn^{+2}S^{+6}O^{-2}4 + K^{+1}2S^{+6}O^{-2}4 + H^{+1}2O^{-2}
}$$
Identifico el medio, cuál se oxida, cuál se reduce y los agentes:

\vspace{0.5\baselineskip}
\medioreduceoxida{Ácido}{Mn}{KMnO4}{O}{H2O2}
\begin{multicols}{2}
Semirreacción de reducción:
$$\left(\ce{
MnO4^- + 8H^+ + 5e^- ->
Mn^{2+} + 4H2O
}\right) \cdot 2$$

Semirreacción de oxidación (identificarla cuesta):
$$\left(\ce{
H2O2 ->
O2 + 2H^+ + 2e^-
}\right) \cdot 5$$
\end{multicols}
    
Las sumo y simplifico:
$$\ce{
2MnO4^- + 16H^+ + 5H2O2 ->
2Mn^{2+} + 8H2O + 5O2 + 10H^+
}$$
$$\ce{
2MnO4^- + 6H^+ + 5H2O2 ->
2Mn^{2+} + 8H2O + 5O2
}$$

Finalmente escribo los coeficientes:
$$\fbox{\ce{2KMnO4 + 5H2O2 + 3H2SO4 -> 5O2 + 2MnSO4 + K2SO4 + 8H2O}}$$


\item Balancear por Redox:
$$\ce{
Zn + HBr + Na3AsO4 -> AsH3 + ZnBr2 + NaBr + H2O
}$$

Escribo sus números de oxidación:
$$\ce{
Zn^0 + H^{+1}Br^{-1} + Na^{+1}3As^{+5}O^{-2}4 ->
As^{-3}H^{+1}3 + Zn^{+2}Br^{-1}2 + Na^{+1}Br^{-1} + H^{+1}2O^{-2}
}$$

\medioreduceoxida{Ácido}{As}{Na3AsO4}{Zn}{Zn}

\begin{multicols}{2}
    Semirreacción de reducción (Arsano no se disocia):
    $$\ce{
    AsO4^{3-} + 11H^+ + 8e^- ->
    AsH3 + 4H2O
    }$$
    
    Semirreacción de oxidación:
    $$\left(\ce{
    Zn ->
    Zn^{2+} + 2e^-
    }\right) \cdot 4$$
\end{multicols}

Las sumo y simplifico
$$\ce{
AsO4^{3-} + 11H^+ + 4Zn ->
AsH3 + 4H2O + 4Zn^{2+}
}$$

Finalmente pongo los coeficientes (NaBr al tanteo):
$$\fbox{\ce{
4Zn + 11HBr + Na3AsO4 ->
AsH3 + 4ZnBr2 + 3NaBr + 4H2O
}}$$

\end{enumerate}


\subsubsection*{Ejercicios del Molinos}

\begin{enumerate}
\item Balancear por Redox:
$$\ce{
HCl + K3PO3 + KMnO4 ->
MnCl2 + KCl + P2O5 + H2O
}$$
Escribo los números de oxidación:
$$\ce{
H^{+1}Cl^{-1} + K^{+1}3P^{+5}O^{-2}3 + K^{+1}Mn^{+7}O^{-2}4 ->
Mn^{+2}Cl^{-1}2 + K^{+1}Cl^{-1} + P^{+5}2O^{-2}5 + H^{+1}2O^{-2}
}$$

\medioreduceoxida{Ácido}{Mn}{KMnO4}{P}{K3PO3}

\begin{multicols}{2}
    Semirreacción de reducción:
    $$\left(\ce{
    MnO4^- + 8H^+ + 5e^- ->
    Mn^{2+} + 4H2O
    }\right) \cdot 4$$
    
    Semirreacción de oxidación:
    $$\left(\ce{
    2PO3^{3-} + 2H^+ ->
    P2O5 + 4e^- + H2O
    }\right) \cdot 5$$
\end{multicols}

Las sumo y simplifico:
$$\ce{
4MnO4^- + 32H^+ + 10 PO3^{3-} + 10H^+ ->
4Mn^{2+} + 16H2O + 5P2O5 + 5H2O
}$$
$$\ce{
4MnO4^- + 42H^+ + 10 PO3^{3-} ->
4Mn^{2+} + 5P2O5 + 21H2O
}$$
Finalmente (El \ce{KCl} al tanteo):
$$\fbox{\ce{
42HCl + 10K3PO3 + 4KMnO4 ->
4MnCl2 + 34KCl + 5P2O5 + 21H2O
}}$$
\end{enumerate}