\section{Balanceo de ecuaciones}

\begin{enumerate}
\item Balancear las siguientes ecuaciones:
\begin{enumerate}
    \item \ce{H2 + O_2 \rightarrow H_2O}

    \item \ce{CH_4 + O_2 \rightarrow H_2O + CO_2}

    \item \ce{N_2 + H_2 \rightarrow NH_3}

    \item \ce{Cr + O_2 \rightarrow Cr_2O_3}

    \item \ce{KClO_3 \rightarrow KCl + O_2}

    \item \ce{BaCl_2 + Na_2SO_4 \rightarrow NaCl + BaSO_4}

    \item \ce{MgS + AlCl_3 \rightarrow MgCl_2 + Al_2S_3}

    \item \ce{Al + H_2SO_4 \rightarrow Al_2(SO_4)_3 + H_2}

    \item \ce{Cu + HNO_3 \rightarrow Cu(NO_3)_2 + NO + H_2O}

    \item \ce{MnO_2 + HCl \rightarrow MnCl_2 +H_2O +Cl_2}

    \item \ce{H_3PO_4 + NO \rightarrow P_2O_3 + HNO_3 +H_2O} %2 2 1 2 2

    \item \ce{MnO_3 + KOH + AuCl_3 \rightarrow AuCl + KCl + KMnO_4 + H_2O} %2 4 1 1 2 2 2

    \item \ce{K2CrO4 + HBr -> CrBr3 + Br2 + KBr + H2O}
\end{enumerate}

\end{enumerate}


\subsection*{Redox}

\begin{enumerate}
\item Balancear las siguientes semireacciones:
\begin{enumerate}
    \item $\ce{ NO_3^- -> N_2 }$ (Medio ácido)
    \item $\ce{ Fe^{2+} -> Fe^{3+} }$
    \item $\ce{ SO_2^{\;2-} -> SO_3^{\;2-} }$ (medio básico)
    \item $\ce{ SO_2^{\;2-} -> SO_3^{\;2-} }$ (medio ácido)
    \item $\ce{ N_2 -> NO_2^- }$ (medio básico)
\end{enumerate}


\item Balancear por Redox:
$$\ce{K2Cr2O7 + KI + H2SO4 ->
K2SO4 + I2 + Cr2(SO4)3 + H2O}$$


\item Balancear por Redox la ecuación que tiene los siguientes reactivos y productos:

\underline{Reactivos}:

permanganato de potasio (KMnO$_4$), hidróxido de potasio (KOH), yoduro de potasio (KI).

\skipline
\underline{Productos}:

Yodato de potasio (KIO$_3$), manganato (VI) de potasio (I) (K$_2$MnO$_4$)


\newpage
\item Balancear por Redox:
$$
\ce{
    K2Cr2O7 + H2SO4 + FeSO4 ->
    Fe2(SO4)3 + Cr2(SO4)3 + H2O + K2SO4
}
$$

Escribo los números de oxidación, identifico cuál se oxida, cuál se reduce, los agentes y el medio:
$$
\ce{
    K^{+1}2Cr^{+6}2O^{-2}7 + H^{+1}2S^{+6}O^{-2}4 + Fe^{+2}S^{+6}O^{-2}4 ->
    Fe^{+3}2(S^{+6}O^{-2}4)3 + Cr^{+3}2(S^{+6}O^{-2}4)3 + H^{+1}2O^{-2} + K^{+1}2S^{-6}O^{-2}4
}
$$

\medioreduceoxida{Ácido}{Cr}{K2Cr2O7}{Fe}{FeSO4}

\begin{multicols}{2}
Semirreacción de reducción:
$$\ce{Cr2O7^{2-} + 14H^+ + 6e^- ->
2Cr^{3+} + 7H2O}$$

Semirreacción de oxidación:
$$\left( \ce{2Fe^{2+} ->
2Fe^{3+} + 2e^-} \right) \cdot 3$$
\end{multicols}

Las sumo:
$$\ce{
    Cr2O7^{2-} + 14H^+ + 6Fe^{2+} ->
    2Cr^{3+} + 7 H2O + 6Fe^{3+}
}$$

Finalmente pongo los coeficientes:
$$\fbox{\ce{
    K2Cr2O7 + 7H2SO4 + 6FeSO4 ->
    3Fe2(SO4)3 + Cr2(SO4)3 + 7H2O + K2SO4
}}$$

\end{enumerate}