\subsection{Ejercicios del Molinos}

\begin{enumerate}
\item Balancear por Redox:
$$\ce{
HCl + K3PO3 + KMnO4 ->
MnCl2 + KCl + P2O5 + H2O
}$$
Escribo los números de oxidación:
$$\ce{
H^{+1}Cl^{-1} + K^{+1}3P^{+3}O^{-2}3 + K^{+1}Mn^{+7}O^{-2}4 ->
Mn^{+2}Cl^{-1}2 + K^{+1}Cl^{-1} + P^{+5}2O^{-2}5 + H^{+1}2O^{-2}
}$$

\medioreduceoxida{Ácido}{Mn}{KMnO4}{P}{K3PO3}

\begin{multicols}{2}
    Semirreacción de reducción:
    $$\left(\ce{
    MnO4^- + 8H^+ + 5e^- ->
    Mn^{2+} + 4H2O
    }\right) \cdot 4$$
    
    Semirreacción de oxidación:
    $$\left(\ce{
    2PO3^{3-} + 2H^+ ->
    P2O5 + 4e^- + H2O
    }\right) \cdot 5$$
\end{multicols}

Las sumo y simplifico:
$$\ce{
4MnO4^- + 32H^+ + 10 PO3^{3-} + 10H^+ ->
4Mn^{2+} + 16H2O + 5P2O5 + 5H2O
}$$
$$\ce{
4MnO4^- + 42H^+ + 10 PO3^{3-} ->
4Mn^{2+} + 5P2O5 + 21H2O
}$$
Finalmente (El \ce{KCl} al tanteo):
$$\fbox{\ce{
42HCl + 10K3PO3 + 4KMnO4 ->
4MnCl2 + 34KCl + 5P2O5 + 21H2O
}}$$


\item Balancear por Redox:
$$\ce{
CrCl3 + KMnO4 + KOH ->
MnCl2 + KCl + K2CrO4 + H2O
}$$

Escribo sus números de oxidación:
$$\ce{
CrCl3 + KMnO4 + KOH ->
MnCl2 + KCl + K2CrO4 + H2O
}$$

\medioreduceoxida{Medio básico}{Mn}{KMnO4}{Cr}{CrCl3}

\begin{multicols}{2}
Semirreacción de reducción:
$$\left(\ce{
MnO4^- + 4H2O + 5e^- ->
Mn^{2+} + 8 OH^{-}
}\right) \cdot 3$$

Semirreacción de oxidación:
$$\left(\ce{
Cr^{3+} + 8OH^- ->
CrO4^{2-} + 4H2O + 3e^-
}\right) \cdot 5$$
\end{multicols}

Las sumo y simplifico:
$$\ce{
3MnO4^- + 12H2O + 5Cr^{3+} + 40OH^- ->
3Mn^{2+} + 24 OH^{-} + 5CrO4^{2-} + 20H2O
}$$
$$\ce{
3MnO4^- + 5Cr^{3+} + 16OH^- ->
3Mn^{2+} + 5CrO4^{2-} + 8H2O
}$$

Finalmente (KCl al tanteo):
$$\fbox{\ce{
5CrCl3 + 3KMnO4 + 16KOH ->
3MnCl2 + 9KCl + 5K2CrO4 + 8H2O
}}$$


\item Balancear por Redox:
$$\ce{KIO3 + H2SO2 + KI ->
I2 + K2SO2 + H2O}$$

Escribo sus números de oxidación:
$$\ce{K^{+1}I^{+5}O^{-2}3 + H^{+1}2S^{+2}O^{-2}2 + K^{+1}I^{-1} ->
I^{0}2 + K^{+1}2S^{+2}O^{-2}2 + H^{+1}2O^{-2}}$$

\medioreduceoxida{Ácido}{I}{KIO3}{I}{KI}

\begin{multicols}{2}
    Semirreacción de reducción:
    $$\ce{
    2IO3^- + 12H^+ + 10e^- ->
    I2 + 6H2O
    }$$
    
    Semirreacción de oxidación:
    $$\left(\ce{
    2I^- ->
    I2 + 2e^-
    }\right) \cdot 5$$
\end{multicols}

Las sumo:
$$\ce{ 2IO3^- + 12H^+ + 10I^- ->
I2 + 6H2O + 5I2 }$$
$$\ce{ 2IO3^- + 12H^+ + 10I^- ->
6I2 + 6H2O }$$

Pongo los coeficientes y simplifico (hay 2 \ce{H^+} por cada \ce{H2SO2}):
$$\ce{ 2KIO3 + 6H2SO2 + 10KI ->
6I2 + 6K2SO2 + 6H2O}$$
$$\fbox{
\ce{ KIO3 + 3H2SO2 + 5KI ->
3I2 + 3K2SO2 + 3H2O}
}$$
\end{enumerate}