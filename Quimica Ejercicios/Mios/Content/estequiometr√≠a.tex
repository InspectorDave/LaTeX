\section{Estequiometría}

\begin{enumerate}
\item La siguiente reacción representa la oxidación del hierro.
Sabiendo que el hierro viene del mineral limonita, el
cual tiene una pureza de \%75. Calcular cuánto óxido férrico
se produjo si en origen se tenían 100 kg de limonita.
$$4\text{Fe} (s) + 3\text{O}_2 (g) \rightarrow 2\text{Fe}_2\text{O}_3 (s)$$

\item Se tiene la oxidación del hierro. Sabiendo que cumple la siguiente ecuación, balancearla. Además, se sabe que se tienen 1.500 g de Fe y la misma masa de O$_2$. Cuál es el reactivo limitante? Cuánta masa de óxido ferroso se forma? Cuánta masa queda sin reaccionar?
$$\text{Fe} + \text{O}_2 \rightarrow \text{Fe}\text{O}$$

\item Teniendo la siguiente fórmula química:
$$\text{Fe} + \text{O}_2 \rightarrow \text{Fe}\text{O}$$
Sabiendo que la reacción tiene un rendimiento del 70\%, cuántos gramos de óxido ferroso se formaron si inicialmente se tenían 700g de hierro?


\item Se tienen 720 gramos de O$_2$, calcular la masa del hidrógeno
necesaria para que reaccione todo el oxígeno y la masa de 
agua formada.

$$\ce{O_2 + 2H_2 -> 2H_2O}$$

Anoto las masas moleculares de cada compuesto:

\hfil$O_2$: 32g/mol\hfil
$H_2$: 2g/mol\hfil
$H_2O$: 18g/mol\hfil

\skipline
Sabiendo que tengo 720g de $O_2$, me fijo cuántos moles son:
\rot{32}{1}{720}{22,5}{g}{mol}

Sabiendo que se tienen 22,5 mol de $O_2$, ahora quiero averiguar cuántos moles de $H_2$ necesito, para lo cual hago regla de 3 simples utilizando los coeficientes estequiométricos de la ecuación:
\rot{1}{2}{22,5}{45}{mol de \ce{O2}}{mol de \ce{H2}}

Hago lo mismo para averiguar la cantidad de $H_2O$ formada:
\rot{1}{2}{22.5}{45}{mol de \ce{O2}}{mol de \ce{H2O}}

Finalmente, averiguo cuánta masa de $H_2$ tengo, utilizando sus moles y su masa molecular:

$45 \text{mol} \cdot 2\text{g/mol} = 90 g$

Para el agua:

$45 \text{mol} \cdot 18\text{g/mol} = 810g$


\item 
Se tienen 200 g de $H_2$ y 200 g de $O_2$.
Definir cuál es el reactivo limitante y cuál está en exceso.
Cuánta masa no reacciona del limitante?
Calcular cuánta agua se forma.

$$O_2 + 2H_2 \longrightarrow 2H_2O$$

Anoto las masas moleculares de cada compuesto:

\hfil$O_2$: 32g/mol\hfil
$H_2$: 2g/mol\hfil
$H_2O$: 18g/mol\hfil

\skipline
Ahora averiguo cuántos moles tengo de cada compuesto:

\hfil$O_2$: 200/32 = 6,25 mol\hfil
$H_2$: 200/2 = 100 mol\hfil

\skipline
Supongo que el oxígeno es el reactivo limitante:
\rot{1}{2}{6,25}{12,5}{\ce{O2}}{\ce{H2}}

Es posible. Ahora analizo el caso en que el hidrógeno es el reactivo limitante:
\rot{2}{1}{100}{50}{mol de \ce{H2}}{mol de \ce{O2}}

No se puede porque tengo menos de 50 moles de $O_2$, por lo tanto habrá $100-12,5 \text{ mol} = 87,5 \text{ mol}$ de $H_2$ que no reaccionen.


\item La ecuación de formación de agua es O$_2$ + H$_2$ $\longrightarrow$ H$_2$O. Si se tienen 640 gramos de oxígeno gaseoso (O$_2$), decir cuántos gramos de hidrógeno gaseoso (H$_2$) se necesitarán y cuántos de agua se formarán. Recordar balancear la ecuación.

\item La ecuación de la formación de óxido de hierro (II) es O$_2$ + Fe $\longrightarrow$ FeO. Si se formaron 900 gramos de óxido, ¿cuántos gramos de hierro y de oxígeno se necesitaron?

\item La ecuación de formación del sulfuro de carbono es C + S$_8$ $\longrightarrow$ CS$_2$. se tienen 1,5 kg de un mineral de carbono de pureza 80\%. ¿Cuánto se formará de sulfuro de carbono?

\item La combustión del butano es CH$_4$ + O$_2$ $\longrightarrow$ H$_2$O + CO$_2$. Si se tiene 1kg de butano diluido con otros gases, teniendo una concentración del 40$\%$, ¿cuántos gramos de agua y de dióxido de carbono se formarán?

\item La combustión del amoníaco es NH$_3$ + O$_2$ $\longrightarrow$ N$_2$ + H$_2$O. Si se tiene un tanque con 5kg de amoníaco con pureza 70\%, decir cuántos moles y cuánta masa de nitrógeno y agua se forman.

\item Teniendo la combustión del hidrógeno O$_2$ + H$_2$ $\longrightarrow$ H$_2$O, se está en un entorno en que dicha reacción tiene un rendimiento del 90\%, decir cuánto se formará de agua y cuánto de oxígeno gaseoso e hidrógeno gaseoso queda sin reaccionar si se tienen 75 kg de O$_2$.

\item Se tiene la descomposición del ozono O$_3$ $\longrightarrow$ O$_2$. Esta reacción tiene un rendimiento del 75\%. Si se tenían 100 g de ozono originalmente, ¿cuántos de oxígeno se formaron y cuánto ozono quedó sin reaccionar?
    
\end{enumerate}
