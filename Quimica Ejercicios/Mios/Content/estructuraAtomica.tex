\section{Estructura de la materia}

\begin{enumerate}
\item Obtener la cantidad de protones de los átomos de los siguientes elementos:
\begin{multicols}{4}
\begin{enumerate}
    \item Oxígeno
    \item Sodio
    \item Helio
    \item Hierro
    \item Oro
    \item Uranio
    \item Calcio
    \item Yodo
\end{enumerate}
\end{multicols}


\item Obtener la cantidad de protones y electrones de los siguientes átomos:
\begin{multicols}{4}
\begin{enumerate}
    \item \ce{Cl}
    \item \ce{Br^-}
    \item \ce{Mg^{2+}}
    \item \ce{S^{2-}}
    \item \ce{H^{+}}
    \item \ce{Co}
    \item \ce{Fe^{3+}}
    \item \ce{K^{+}}
\end{enumerate}    
\end{multicols}


\item Obtener la cantidad de protones, neutrones, electrones, Z y A de los siguientes átomos:
\begin{multicols}{4}
\begin{enumerate}
    \item \ce{^{36}_{17}Cl}
    \item \ce{^{35}_{17}Cl^-}
    \item \ce{^{37}_{19}K^+}
    \item \ce{^{235}_{92}U}
    \item \ce{^{63}Cu^{2-}}
    \item \ce{^{238}U^{3+}}
    \item \ce{^{79}Se^{2-}}
    \item \ce{^{134}Cs^{2+}}
\end{enumerate}    
\end{multicols}


\item Dar el símbolo químico y su cantidad de protones, neutrones y electrones de: 
\begin{enumerate}
    \item Un ion divalente metálico isoelectrónico con el argón.

    \item Un ion alcalino monovalente del tercer período.

    \item Un anión trivalente isoelectrónico con el kriptón.

    \item Un catión alcalinotérreo isoelectrónico con el \ce{S^{2-}}.
\end{enumerate}


\item Calcular la masa de las siguientes cantidades:
\begin{enumerate}
    \item 8 moles de óxido de ferroso (FeO) %576g

    \item 40 moles de agua (\ce{H2O}) %720g

    \item Medio mol de óxido de azufre (\ce{S2O}) %40g
\end{enumerate}


\item Calcular la cantidad de moles de las siguientes masas:
\begin{enumerate}
    \item 150 gramos de gas oxígeno (\ce{O2}) % 4,69mol

    \item 400 gramos de nitrato de potasio (\ce{KNO3}) % 3,96mol

    \item 20 gramos de gas nitrógeno (\ce{N2}) % 0,71mol
\end{enumerate}


\item Obtener la masa atómica relativa de los siguientes elementos:
\begin{enumerate}
    \item Se tiene en la naturaleza $^{63}$Cu y $^{65}$Cu. El más ligero representa el 69,17\% de los átomos de cobre encontrados en la naturaleza y el resto pertenece al más pesado.

    \item Se tiene en la naturaleza $^{235}$U y $^{238}$U. Se sabe que la concentración del primer isótopo es del 95\% y la del segundo del 5\%.

    \item Se tienen dos isótopos de cloro en la naturaleza: Cl$^{35}$ y Cl$^{37}$. Su porcentaje de aparición es 75,7\% y 24,3\% respectivamente.
\end{enumerate}


\item Obtener los porcentajes de aparición de cada isótopo a partir de las siguientes descripciones:
\begin{enumerate}
    \item Sabiendo que la MAR del Cobre es 63,54 y que tiene dos isótopos en la naturaleza, Cu$^{63}$ y Cu$^{65}$.

    \item El elemento manganeso presenta 3 isótopos de número másico correlativo, teniendo el más pesado A=55, el pocentaje de los dos primeros es igual y la masa atómica relativa es de 54,954.
\end{enumerate}


\item A partir de los siguientes elementos ordenarlos según su:

\hfil$\ce{Ni, Al, Si, Li, Cl, Mg, H, F, K, Cs}$\hfil

\begin{enumerate}
    \item Radio atómico.
    \item Energía de ionización.
    \item Afinidad electrónica.
    \item Electronegatividad.
\end{enumerate}


\item Decir las geometrías moleculares de las siguientes moléculas:
\begin{multicols}{2}
\begin{enumerate}
    \item $\ce{H2O}$
    \item $\ce{CO}$
    \item $\ce{CO2}$
    \item $\ce{NH3}$
    \item $\ce{CF4}$
    \item $\ce{O3}$
    \item $\ce{SO2}$
    \item $\ce{O2}$
    \item $\ce{BF3}$
    \item[]
\end{enumerate}
\end{multicols}


\item Ordenar las siguientes sustancias por punto de ebullición:
$$\ce{H2O, N2, NaCl, SO}$$


\end{enumerate}
