\section{Estructura de la materia}

\begin{enumerate}
\item Tengo 8 moles de FeO (óxido de hierro) ¿cuánto pesan? 
%576g
\item Tengo 40 moles de agua (H$_2$O), ¿cuánto pesan? 
%720g
\item Tengo medio mol de óxido de azufre (S$_2$O), ¿cuánto pesa? 
%40g
\item ¿Cuántos moles son 150 gramos de gas oxígeno (O$_2$)? 
%4,69mol
\item ¿Cuántos moles son 400 gramos de nitrato de potasio (KNO$_3$)?
%3,96mol
\item ¿Cuántos moles son 20 gramos de gas nitrógeno (N$_2$)?
%0,71mol
\end{enumerate}

\subsubsection*{Masa atómica relativa}

\begin{enumerate}
\item Se tiene en la naturaleza $^{63}$Cu y $^{65}$Cu. El más ligero representa el \%69,17 de los átomos de cobre encontrados en la naturaleza y el resto pertenece al más pesado. ¿Cuál es la MAR?

\item Se tiene en la naturaleza $^{235}$U y $^{238}$U. Se sabe que la concentración del primer isótopo es del 95\% y la del segundo del 5\%. Calcular la masa atómica relativa.

\item Calcular la MAR sabiendo que se tienen dos isótopos de cloro en la naturaleza: Cl$^{35}$ y Cl$^{37}$. Su porcentaje de aparición es 75,7\% y 24,3\% respectivamente.

\item Sabiendo que la MAR del Cobre es 63,54 y que tiene dos isótopos en la naturaleza, Cu$^{63}$ y Cu$^{65}$, calcular el porcentaje de aparición de cada uno.
\end{enumerate}


\subsubsection*{Propiedades de los elementos}

\begin{enumerate}
    \item Teniendo los siguientes elementos:
    
    \hfil$\ce{Ni, Al, Si, Li, Cl, Mg, H, F, K, Cs}$\hfil

    Ordenelos según su:

    \begin{enumerate}
        \item Radio atómico.
        \item Energía de ionización.
        \item Afinidad electrónica.
        \item Electronegatividad.
    \end{enumerate}
\end{enumerate}
