\section{Equilibrio}

\begin{enumerate}
    \item Se dispone de 15 L de una SC de ácido sulfhídrico, pKa = 4 y pH = 3,7. Calcular la cantidad de ácido en el equilibrio.
    \item Se dispone de 20 L de una SC de ácido selenhídrico, pKb = 11,5 y pOH = 12,3. Calcular la cantidad de ácido y la cantidad de la base conjugada en el equilibrio.
    \item Se tienen 750ml de una solución de hidróxido de aluminio. Sabiendo que el pH=12 y pKa=11,5, calcular la cantidad de la base en el equilibrio.
    \item Calcular el pOH de una solución de ácido sulfhídrico cuya concentración es $0,0625 \%m/V$ y $K_a = 6,95 \times 10^{-4}$.
\end{enumerate}