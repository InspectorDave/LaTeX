\section{Calorimetría}

\begin{enumerate}
\item Calcular el calor necesario para aumentar la temperatura de 700g de agua desde 10ºC a 25ºC.

\item Un vaso de agua que contiene 200g pasó de estar a 20ºC a 5ºC. Cuánto calor ganó/perdió?

\item Una lamina de hierro que está a 20ºC se calienta hasta estar a 200ºC. Tiene una masa de 60kg. Calcular el calor necesario. $C_{\text{Fe}}=0,107 \dfrac{\text{cal}}{\text{g} \cdot \text{ºC}}$

\item Se mezclan 100 g de agua a 20ºC con 50 g de agua a 90ºC. Calcular la temperatura final de la mezcla.

\item Se mezclan 450 g de agua a 5ºC con 40 g de hierro a 200ºC. Calcular la temperatura final de la mezcla.

\item Calcular la energía necesaria para fundir 700 g de hielo que están a 0º C.

\item Calcular el calor necesario para evaporar 4 kg de agua líquida que está a 100ºC.

\item Calcular el calor necesario para calentar 2 toneladas de hielo desde 100 K hasta -10 ºC.

\item Calcular el calor necesario para calentar 1500 g toneladas de vapor desde 110 ºC hasta 115 ºC.

\item Calcular el calor necesario para a partir de 150 g de agua líquida a 70ºC obtener vapor de agua a 200ºC.

\item Calcular el calor necesario para obtener agua a 30 ºC a partir de 200 g de hielo a 250 K.

\item Calcular el calor necesario para calentar 200 g de hielo a -15ºC hasta 250 ºC de vapor de agua.

\item Calcular la energía necesaria para elevar 38,6 kg de agua desde 67,5 ºC hasta 97ºC.

\item Se tiene un balde con 500g de agua. Se sabe que inicialmente estaba a 20ºC y al final del día a 25ºC. Calcular su variación de energía.

\item Se tienen 200 g de hielo a -9ºC. Calcular cuánto calor se necesita para elevar la temperatura de ese agua a 130ºC.

\item Se tienen 700 g de agua a 70ºC y se le añaden 150 g de agua a 25ºC. Calcular la temperatura final de la mezcla.

\item Se tienen 0,8 kg de agua a 20ºC y se le añaden 100 g de agua a 97ºC. Calcular la temperatura final de la mezcla.

\item Se tienen 500 g de agua a 45ºC y se le añaden 200 g de alcohol etílico (c = 0,41$\frac{\text{cal}}{\text{g} \cdot \text{ºC} }$) a 3ºC. Calcular la temperatura final de la mezcla.

\item En un caluroso día de verano se tiene una jarra de 2,5 L de agua a 45ºC y se le añaden 100 g de cubitos de hielo a -10ºC. Calcular la temperatura final de la mezcla.

\item A 300 g de agua a 70ºC se le añaden 200 g de hielo a -5ºC. Calcular la temperatura final de la mezcla.

\end{enumerate}

\subsection*{Pasaje de unidades en temperaturas}

\begin{enumerate}
\item Pasar las siguientes temperaturas de $^\circ$C a $^\circ\text{F}: -50, -10, -42, 0, 50, 100, 500, -40$.

\item Pasar las siguientes temperaturas de $^\circ$F a $^\circ\text{C}: -50, -10, -42, 0, 50, 100, 500, -40$.

\item Pasar las siguientes de K a $^\circ$C: $0, 50, 100, 273, 1000, -50$.

\item Pasar las siguientes de $^\circ$C a K: $-100, 0, 50, 100, 273, 1000, -50$.

\item ¿A qué temperatura $\text{$^\circ$C}$ es igual a ºF?
\begin{align*}
    ^\circ\text{F}&= \text{$^\circ$C} \cdot 1,8 + 32\\
    T &= T \cdot 1,8 + 32\\
    -32 &= 0,8 T\\
    \dfrac{-32}{0,8} &= T\\
    T &= -40
\end{align*}

\item ¿A qué temperatura $\text{$^\circ$F}$ es igual a K?
\begin{align*}
    ^\circ\text{F}&= \text{$^\circ$C} \cdot 1,8 + 32\\
    ^\circ\text{F}&= (\text{K}-273) \cdot 1,8 + 32\\
    T&= (T-273) \cdot 1,8 + 32\\
    T &= 574,59
\end{align*}
\end{enumerate}
