\section{Calorimetría}

\begin{enumerate}
\item Cuánto calor es necesario para aumentar en 15ºC la temperatura de 700g de agua?

\item Un vaso de agua que contiene 200g pasó de estar a 20ºC a 5ºC. Cuánto calor ganó/perdió?

\item Una lamina de hierro que está a 20ºC se calienta hasta estar a 200ºC. Tiene una masa de 60kg. Cuánto calor se necesitó? $C_{\text{Fe}}=0,107 \dfrac{\text{cal}}{\text{g} \cdot \text{ºC}}$

\item Se mezclan 100 g de agua a 20ºC con 50 g de agua a 90ºC. Cuál es la temperatura final de la mezcla?

\item Calcular la energía necesaria para fundir 700 g de hielo que están a 0º C.

\item Calcular el calor necesario para evaporar 4 kg de agua líquida que está a 100ºC.

\item Calcular el calor necesario para calentar 2 toneladas de hielo desde 100 K hasta -10 ºC.

\item Calcular el calor necesario para calentar 1500 g toneladas de vapor desde 110 ºC hasta 115 ºC.

\item Calcular el calor necesario para a partir de 150 g de agua líquida a 70ºC obtener vapor de agua a 200ºC.

\item Calcular el calor necesario para obtener agua a 30 ºC a partir de 200 g de hielo a 250 K.

\item Calcular el calor necesario para calentar 200 g de hielo a -15ºC hasta 250 ºC de vapor de agua.

\item Calcular la energía necesaria para elevar 38,6 kg de agua desde 67,5 ºC hasta 97ºC.

\item Se tiene un balde con 500g de agua. Se sabe que inicialmente estaba a 20ºC y al final del día a 25ºC. Calcular su variación de energía.

\item Se tienen 200 g de hielo a -9ºC. Calcular cuánto calor se necesita para elevar la temperatura de ese agua a 130ºC.
\end{enumerate}

\subsection*{Pasaje de unidades en temperaturas}

\begin{enumerate}
\item Pasar las siguientes temperaturas de $^\circ$C a $^\circ\text{F}: -50, -10, -42, 0, 50, 100, 500, -40$.

\item Pasar las siguientes temperaturas de $^\circ$F a $^\circ\text{C}: -50, -10, -42, 0, 50, 100, 500, -40$.

\item Pasar las siguientes de K a $^\circ$C: $0, 50, 100, 273, 1000, -50$.

\item Pasar las siguientes de $^\circ$C a K: $-100, 0, 50, 100, 273, 1000, -50$.

\item ¿A qué temperatura $\text{$^\circ$C}$ es igual a ºF?
\begin{align*}
    ^\circ\text{F}&= \text{$^\circ$C} \cdot 1,8 + 32\\
    T &= T \cdot 1,8 + 32\\
    -32 &= 0,8 T\\
    \dfrac{-32}{0,8} &= T\\
    T &= -40
\end{align*}

\item ¿A qué temperatura $\text{$^\circ$F}$ es igual a K?
\begin{align*}
    ^\circ\text{F}&= \text{$^\circ$C} \cdot 1,8 + 32\\
    ^\circ\text{F}&= (\text{K}-273) \cdot 1,8 + 32\\
    T&= (T-273) \cdot 1,8 + 32\\
    T &= 574,59
\end{align*}
\end{enumerate}


\subsection*{Dilatación térmica}
