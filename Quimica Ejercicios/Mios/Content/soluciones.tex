\section{Soluciones}

\subsubsection*{Ejercicios básicos de soluciones:}

\begin{enumerate}
\item Se tienen 1000 g de agua salada, se sabe que hay 50 g de sal. Calcular el $\dfrac{\%\text{m}}{\text{m}}$ de la solución.


\item Se tienen 2 kg de agua, al cual se le agregan 0,5 kg de azúcar.
\begin{enumerate}
    \item Calcular la masa de la solución.
    \item Calcular el $\dfrac{\%\text{m}}{\text{m}}$ de la solución. 
\end{enumerate}


\item Se tienen 80 g de nitrato de sodio disueltos en 1 kg de agua. Calcular el $\dfrac{\%\text{m}}{\text{v}}$ de la solución.


\item Se tienen 2 kg de nitrato de potasio disuelto en 10 l de glicerol. 
\begin{enumerate}
    \item Nombrar cuál es la solución, cuál el solvente y cuál el soluto.

    \item Calcular el $\dfrac{\%\text{m}}{\text{v}}$ de la solución.
\end{enumerate}


\item Se tienen 8 l de agua y 2 l de alcohol etílico. 
\begin{enumerate}
    \item Nombrar cuál es la solución, cuál el solvente y cuál el soluto; además decir cuál es el volumen de cada uno.
    
    \item Calcular el $\dfrac{\%\text{v}}{\text{v}}$ de la solución.
\end{enumerate}


\item Se tienen 8 l de agua y 12 l de alcohol metílico. 
\begin{enumerate}
    \item Nombrar cuál es la solución, cuál el solvente y cuál el soluto; además decir cuál es el volumen de cada uno.

    \item Calcular el $\dfrac{\%\text{v}}{\text{v}}$ de la solución.
\end{enumerate}


\item Se tiene un lingote de bronce de 800 g. De esos 800 g, hay 140 g de estaño. Decir cuál es la solución, cuál el soluto y cuál el solvente. ¿Cuál es el \% m/m? % 17,5 %

\item Se tienen 2 litros de agua. La masa de la solución agua salada es de 2,2 kg. Cuál es el \%m/m? % 9,09 %

\item Se tienen 200 g de azúcar disueltos en 1,5 kg de alcohol etílico. Cuál es el \%m/m? % 11,8

\item Un lingote de oro de 12,4kg dice tener 0,2 \%m/m de plata. Cuál es la masa de plata que hay en el lingote?

\item Tengo una botella de 750 ml de vodka, se sabe que tiene medio litro de agua. Cuál es el \% V/V del alcohol etílico?

\item Se tiene una botella de vino de 2,25 l. Dice tener un \%V/V de 12,5\%. Cuál es el volumen de soluto y cuál el de solvente?

\item Se tienen 0,3 kg de azúcar disueltos en 500 ml de agua. Calcular \%m/m y \%m/V.

\item Hay 4.500 cm$^3$ de agua salada. Se sabe que en ese agua salada hay 250 g de NaCl. Calcular la M de la solución. %RTA: 0,95 M

\item Se tiene un lingote de latón, que contiene 5.600 g de cobre y 2.400 g de zinc. Calcular la m de la solución. %RTA: 6,56 m

\item Se tiene una solución formada por 8 l de agua con 2 l de alcohol ($\delta_{\text{C}_2\text{H}_6\text{O}}=0,79 g/ml$). Calcular \%m/m, \linebreak
\%V/V, \%m/V, M y m.

\item Se tienen 3 litros de una solución de agua salada ($\delta = 1,05$g/ml). Sabiendo que hay 200g de sal disueltos, calcular \%m/m y \%m/v. %Rta:  %m/m: 6,35, %m/v: 6,66%

\item Se tienen 100 ml de una solución 0,2m de NaCl ($\delta = 1,2$ g/ml). Calcular masa de soluto. %No hace falta hacer el sistema de ecuaciones. Poner que la molalidad es esa cantidad de moles en 1kg de solvente más molalidad por masamolar de st, que sería masa de solución en kg. Rta: 1,39

\item Se tiene una solución de KCl 0,1m y $\delta=1,08$ g/ml. Expresar su M, \%m/v y \%m/m.

\item Se tiene una solución de \ce{MgCl2} 0,2m y $\delta=1,1$ g/ml. Expresar su M, \%m/v y \%m/m.
\end{enumerate}

\subsubsection*{pH}

\begin{enumerate}
\item Se tiene una solución de 3 litros con 5 mg de HNO$_3$. 
Calcular el pH de la solución.
\begin{align*}
\text{1 mol de HNO}_3 \;&-\; 63 g\\
 79,3\times 10^{-6} \;&-\; 0,005g
\end{align*}

$$[H^+] =  \dfrac{79,3\times 10^{-6}}{3l} = 26,4 \times 10^{-6}$$

$$pH=-\log([H^+]) = 4,57$$

\item Se tiene una solución de medio litro con 30 mg de HClO$_3$. Calcular el pH de la solución.
\begin{align*}
\text{1 mol de HClO}_3 \;&-\; 84,4 g\\
 3,55\times 10^{-4} \;&-\; 0,03g
\end{align*}

$$[H^+] =  \dfrac{3,55\times 10^{-4}}{0,5l} = 
7,1 \times 10^{-4}$$

$$pH=-\log([H^+]) = 3,15$$

\skipline
\item Se tiene una solución de 300 ml con 5 mg de ácido sulfhídrico (H$_2$S). Calcular su pH y pOH.
\begin{align*}
\text{1 mol de H}_2\text{S} \;&-\; 34 g\\
 1,47\times 10^{-4} \;&-\; 0,005g
\end{align*}

\skipline

$$[H^+] =  \dfrac{ 2\cdot 1,47\times 10^{-4}}{0,3l} = 
9,8 \times 10^{-4}$$

$$pH=-\log([H^+]) = 3,01$$

\skipline
\item Se tiene una solución de 750 ml con 10 mg de Co(OH)$_3$. Calcular su pH y pOH.
\begin{align*}
\text{1 mol de Co(OH)}_3 \;&-\; 110 g\\
 9,09\times 10^{-5} \;&-\; 0,01g
\end{align*}

\skipline

$$[(OH)^-] =  \dfrac{ 3\cdot  9,09\times 10^{-5}}{0,75l} = 
3,63 \times 10^{-4}$$

$$pOH=-\log([(OH)^-]) = 3,44
\Rightarrow pH = 10,56$$

\skipline
\item Se tiene una solución de 100 l con 1 mg de HI. Calcular su pH y pOH.
\begin{align*}
\text{1 mol de HI} \;&-\; 128 g\\
 7,81\times 10^{-6} \text{ mol de HI} \;&-\; 0,001g
\end{align*}
\end{enumerate}
