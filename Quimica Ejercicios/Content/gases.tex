\section{Gases ideales}

\begin{enumerate}
\item Se tiene en un contenedor de 5 litros de volumen un gas a 20ºC que está a una presión de 14 atm. El contenedor se calienta hasta 25ºC y se dilata medio litro. Cuál es la presión final?

\item Se tienen 800g de O$_2$, en un recipiente de 2l, a una presión de 6 atm. Calcular la temperatura del gas.

\item La presión final es la mitad de la inicial. Sabiendo que la temperatura inicial es 27 $^\circ$C y que el volumen se triplica, cuál es la temperatura final?

\item Se tienen 5 kg de hidrógeno gaseoso. Cuántos moles son?

\item Se tiene dióxido de carbono en condiciones normales de presión y temperatura (CNPT) en un container de 50 l. Cuántos moles se tienen? Y cuánto pesan?

\item Se tiene un recipiente con una solución de gases adentro. El recipiente es de 100 litros, la presión es 82 atm y la temperatura es 400 K. Sabiendo que de todos los moles del recipiente, la mitad es de oxígeno gaseoso y la otra mitad de gas nitrógeno, cuánta masa hay de cada gas? Y cuántas moléculas?
\end{enumerate}


\subsubsection*{Ley de Henry (Gases disueltos en líquido)}


\section{Hidroestática}


\subsubsection*{Presión}


\subsubsection*{Variación de la presión con la profundidad} 


\subsubsection*{Principio de Arquímides}
