\section{Balanceo de ecuaciones}


\subsection{Método algebraico}

\underline{Ejemplo:}

$$\text{Fe} + \text{HCl} \longrightarrow \text{FeCl}_3 + \text{H}_2$$

\begin{minipage}[l]{0.2\textwidth}
\begin{align*}
\text{Fe}&: A = C\\
\text{H}&: B = 2D\\
\text{Cl}&: B = 3C\\
\;
\end{align*}
\end{minipage}
$\Longrightarrow$
\hfil
$B=2D=3C=3A$
\hfil
$\Longrightarrow$
\begin{minipage}[l]{0.1\textwidth}
\begin{align*}
A&= 2\\
B&= 6\\
C&= 2\\
D&= 3
\end{align*}
\end{minipage}
$\Longrightarrow$
\hfil 
\fbox{$2\text{Fe} + 6\text{HCl} \longrightarrow 2\text{FeCl}_3 + 3\text{H}_2$}
\hfil


\subsection{Redox}

\begin{enumerate}
    \item $NO_3^- \rightarrow N_2$ (Medio ácido)
    \item $Fe^{2+} \rightarrow Fe^{3+}$
    \item $SO_2^{\;2-} \rightarrow SO_3^{\;2-}$ (medio básico)
    \item $SO_2^{\;2-} \rightarrow SO_3^{\;2-}$ (medio ácido)
    \item $N_2 \rightarrow NO_2^-$ (medio básico)
\end{enumerate}


\subsubsection*{Reacción redox}

\begin{enumerate}

\item Balancear por redox:
$$3\text{As}_2\text{O}_3 + 4\text{HNO}_3 + 7\text{H}_2\text{O} \rightarrow
6\text{H}_3 \text{AsO}_4 + 4\text{NO}$$

Se ve que As se oxida y que N se reduce.

\begin{multicols}{2}

\underline{Semirreacción de oxidación:}
$$\text{As}_2\text{O}_3 + 5\text{H}_2\text{O}\longrightarrow
2(\text{AsO}_4)^{3-} + 10\text{H}^+ + 4e^-$$

\underline{Semirreacción de reducción:}
$$(\text{NO}_3)^{-} + 4 \text{H}^+ \longrightarrow
\text{NO} + 2\text{H}_2\text{O} - 3e^-$$
\end{multicols}

Ahora multiplico la semirreacción de oxidación por 3 y la de reducción por 4, de manera que al sumarlas se cancelen los $e^-$:

\begin{multicols}{2}

\underline{Semirreacción de oxidación:}
$$\big(\text{As}_2\text{O}_3 + 5\text{H}_2\text{O}\longrightarrow
2(\text{AsO}_4)^{3-} + 10\text{H}^+ + 4e^-\big) \cdot 3$$

\underline{Semirreacción de reducción:}
$$\big((\text{NO}_3)^{-} + 4 \text{H}^+ \longrightarrow
\text{NO} + 2\text{H}_2\text{O} - 3e^-\big) \cdot 4$$
\end{multicols}


\item Balancear por Redox:
$$\text{K}_2\text{Cr}_2\text{O}_7 + \text{KI} + \text{H}_2\text{SO}_4 \longrightarrow \text{K}_2\text{SO}_4 + \text{I}_2 + \text{Cr}_2(\text{SO}_4)_3 + \text{H}_2\text{O}$$


\item 
Balancear la ecuación que tiene los siguientes reactivos y productos:

Reactivos:

permanganato de potasio (KMnO$_4$), hidróxido de potasio (KOH), yoduro de potasio (KI).

\skipline
Productos:

Yodato de potasio (KIO$_3$), manganato (VI) de potasio (I) (K$_2$MnO$_4$)

\skipline
{\large
\hfil
K$^{+1}$Mn$^{+7}$O$_4^{-2}$ \hfil+\hfil
K$^{+1}$O$^{-2}$H$^{+1}$ \hfil+\hfil
K$^{+1}$I$^{-1}$ \hfil$\longrightarrow$\hfil
K$^{+1}$I$^{+5}$O$_3^{-2}$ \hfil+\hfil
K$_2^{+1}$Mn$^{+6}$O$_4^{-2}$
\hfil
}
\end{enumerate}


\subsubsection*{\underline{Ejemplo simple:}}

$$\text{Fe} + \text{Cl}_2 \longrightarrow \text{FeCl}_2$$

\begin{multicols}{2}

\underline{Semirreacción de oxidación:}

$$\text{Fe} - 2e^- \longrightarrow \text{Fe}^{2+}$$

\underline{Semirreacción de reducción:}

$$\text{Cl}_2 + 2e^- \longrightarrow 2\text{Cl}^{-} $$

\end{multicols}

Las reacciones Redox pueden ser en medios ácidos o básicos. En caso de ser medio ácido, hay que compensar la falta de hidrógenos con cationes H$^+$. En caso de ser medio básico esta falta se compensa con grupos hidroxilos (OH)$^-$.


\skipline
\textbf{\underline{Ejemplo medio ácido:}}

$$\text{KI} + \text{K} \text{I} \text{O}_3 + \text{H} \text{Cl} \longrightarrow
\text{KCl} + \text{I}_2 +  \text{H}_2 \text{O}$$

\hfil Ioduro de potasio \hfil + \hfil iodato de potasio \hfil + \hfil ácido clorhídrico \hfil
$\rightarrow$ \hfil
cloruro de potasio \hfil + \hfil iodo molecular \hfil + \hfil agua \hfil 

\begin{multicols}{2}

\underline{Semirreacción de oxidación:}

$$2\text{I}^{-} \longrightarrow \text{I}_2 + 2e^-$$

\underline{Semirreacción de reducción:}

$$2(\text{IO}_3)^{-} + 12 \text{H}^+ \longrightarrow \text{I}_2 + 6\text{H}_2 \text{O} - 10e^-$$
\end{multicols}

Multiplico para que la cantidad de electrones sean iguales (multiplicar por cinco la de oxidación) y las sumo:

$$10\text{I}^{-} + 2(\text{IO}_3)^{-} + 12 \text{H}^+ \longrightarrow 5 \text{I}_2 + 10 e^- + \text{I}_2 + 6\text{H}_2 \text{O} - 10e^-$$

Cancelo los electrones, y si puedo simplifico, quedando:

$$5\text{I}^{-} + (\text{IO}_3)^{-} + 6 \text{H}^+ \longrightarrow 3 \text{I}_2 + 3\text{H}_2 \text{O}$$

Finalmente pongo los coeficientes en la fórmula original y agrego lo que haga falta:

$$5\text{KI} + \text{K} \text{I} \text{O}_3 + 6\text{H} \text{Cl} \longrightarrow
6\text{KCl} + 3\text{I}_2 +  3\text{H}_2 \text{O}$$

