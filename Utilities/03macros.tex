\makeatletter
\newcommand*\bigcdot{\mathpalette\bigcdot@{.5}}
\newcommand*\bigcdot@[2]{\mathbin{\vcenter{\hbox{\scalebox{#2}{$\m@th#1\bullet$}}}}}
\makeatother
\newcommand{\skipline}{\vspace{\baselineskip}}
\newcommand{\sskip}{\vspace{0.5\baselineskip}} % Small skip
\newcommand{\tab}{\hspace{8mm}}

\newcommand{\R}{\mathbb{R}}
\newcommand{\knn}{\mathbb{K}^{n \times n}}
\newcommand{\kn}{\mathbb{K}^{n}}
\newcommand{\rnn}{\mathbb{R}^{n \times n}}
\newcommand{\rmn}{\mathbb{R}^{m \times n}}
\newcommand{\rn}{\mathbb{R}^{n}}
\newcommand{\cnn}{\mathbb{C}^{n \times n}}
\newcommand{\cmn}{\mathbb{C}^{m \times n}}
\newcommand{\cn}{\mathbb{C}^{n}}
\newcommand{\A}{\textgoth{A}}

\newcommand{\ber}{\mathbb{1}}

\newcommand{\fl}{\hspace*{\fill}\\}

\newcommand{\ssmall}{\footnotesize}
\newcommand{\sssmall}{\scriptsize}

\newcommand{\rot}[6]{ % Rule Of Three
\begin{align*}
    #1\text{ #5 } &\rule[0.5ex]{3em}{0.5pt} \ #2\text{ #6}\\
    #3\text{ #5 } &\rule[0.5ex]{3em}{0.5pt} \ #4\text{ #6}
\end{align*}
}

\newcommand{\medioreduceoxida}[5]{
    \begin{table}[H]
    \centering
    \renewcommand{\arraystretch}{1.3}
    \begin{tabular}{p{0.2\textwidth}p{0.2\textwidth}p{0.3\textwidth}}
        \multirow{2}{*}{Medio #1} & Se reduce: \ce{#2} & Agente oxidante: \ce{#3} \\
        & Se oxida: \ce{#4} & Agente reductor: \ce{#5}
    \end{tabular}
    \end{table}
}
