\section*{Resolución}

\noindent
\begin{enumerate}[left=0cm]
\item Algunas aclaraciones:
\end{enumerate}

\begin{itemize}
\item \#Direcciones = \#Hosts + \#Routers + 2
\item Los enlaces entre routers también tienen su subred.
\item Los tamaños de bloque son potencias de 2.
\item Las redes se asignan de mayor tamaño de bloque a menor.
\end{itemize}

Será útil ir contando cuántos hosts, routers y direcciones necesitamos para cada subred, así que se arma la siguiente tabla y se la llena. También puede ayudar tener al costado de la tabla los bits utilizables de las direcciones (en rojo, los bits que son parte del prefijo de subred):

\begin{table}[H]
    \centering
    \begin{tabular}{|c|r|r|r|r|c|c}
    \cline{1-6}
    \multicolumn{1}{|c|}{Subnet} & \multicolumn{1}{c|}{\#Hosts} & \multicolumn{1}{c|}{\#Routers} & \multicolumn{1}{c|}{\#Direcciones} & \multicolumn{1}{c|}{Tamaño bloque} & \multicolumn{1}{c|}{Prefijo de subred} & Última parte\\ \cline{1-6}
    C & 500 & 1 & 503 & 512 & 195.42.40.0/23 & {\color{red}0}0.00000000\\ \cline{1-6}
    A & 126 & 1 & 129 & 256 & 195.42.42.0/24 & {\color{red}10}.00000000\\ \cline{1-6}
    B & 61 & 1 & 64 & 64 & 195.42.43.0/26 & {\color{red}11.00}000000\\ \cline{1-6}
    D & 50 & 1 & 53 & 64 & 195.42.43.64/26 & {\color{red}11.01}000000\\ \cline{1-6}
    E & 30 & 1 & 33 & 64 & 195.42.43.128/26 & {\color{red}11.10}000000\\ \cline{1-6}
    p & 0 & 3 & 5 & 8 & 195.42.43.192/29 & {\color{red}11.11000}000\\ \cline{1-6}
    F & 1 & 1 & 4 & 4 & 195.42.43.200/30 & {\color{red}11.110010}00\\ \cline{1-6}
    $\text{r}_{34}$ & 0 & 2 & 4 & 4 & 195.42.44.204/30 & {\color{red}11.110011}00\\\cline{1-6}
    \end{tabular}
\end{table}

\begin{enumerate}[start=2, left=0cm]
    \item ¿Por qué se suma 2 a la cantidad de direcciones? Son la dirección de la subred (la primera, todos 0) y la dirección de broadcast (la última, todos 1).

    \item ¿Por qué los tamaños de bloque son potencias de 2? Porque la máscaras de red son bits seguidos de 1.

    \item ¿Cuántas direcciones sin utilizar quedaron?
    
    \hfil$1024 - 503 - 129 - 64 - 53 - 33 - 5 - 4 - 4 = 229$\hfil

    \item De las direcciones sin utilizar ¿Cuántas son útiles?
    
    \hfil$1024 - 512 - 256 - 64 - 64 - 64 - 8 - 4 - 4 = 48$\hfil

    \item ¿Se puede agregar una subred de 200 hosts? Es posible, si el router de esa subred tiene NAT activado.

\end{enumerate}