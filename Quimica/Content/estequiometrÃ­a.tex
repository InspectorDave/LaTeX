\section{Estequiometría}
\subsubsection*{Cómo resolver un ejercicio básico de estequiometría:}

\begin{itemize}
    \item Balancear la ecuación.
    \item Ver cuánto pesa un mol de cada compuesto (cuánto pesa un mol de agua, de gas oxígeno, de cloruro de sodio, etc).
    \item Ver qué información tengo y utilizando la regla de 3 simple averiguar los gramos y moles que tengo de cada compuesto.
    \item Recordar que haber balanceado la ecuación nos dice la relación entre la cantidad de moles de cada compuesto.
\end{itemize}

\subsection*{Constante de equilibrio}

La constante de equilibrio de una reacción con rendimiento menor a 100\%, será la siguiente:
$$\ce{
aA + bB \rightleftharpoons cC + dD
}$$
$$
K_C = \dfrac{[C]^a \cdot [D]^c}{[A]^a\cdot [B]^b}
$$