\section{Electroquímica}

La fuerza electromotriz (FEM, $\bigepsilon$) de una pila se calcula en base a los potenciales de reducción de las reacciones que ocurren (están tabulados).
$$\Delta\bigepsilon = \bigepsilon_{\text{cátodo}} - \bigepsilon_{\text{ánodo}}$$

El cátodo se reduce, el ánodo se oxida.

Si $\Delta\bigepsilon>0$ la reacción es espontánea. Si no, hay que invertir reactivos con productos.

\subsubsection*{Condiciones no normales}

Ecuación de Nerst: 
$$\bigepsilon = \bigepsilon_0  - \dfrac{R\cdot T}{n \cdot F} \cdot \left( \dfrac{[\text{productos}]}{[\text{reactivos}]} \right)$$
