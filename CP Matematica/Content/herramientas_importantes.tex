\section{Herramientas importantes}

\subsection{Completar de cuadrados}

En algunos casos, se pueden tener expresiones que se parecen a un trinomio cuadrado perfecto ($A^2 + 2AB + B^2$), pero no lo son. Es posible ``crearlo'' siguiendo algunos pasos y luego se debería poder resolver el ejercicio.

Se suelen hacer los siguientes pasos:
\begin{enumerate}
    \item Identificar a $A$.
    \item Identificar lo que sería $2AB$, de tal manera de luego despejar cuánto debería valor $B$.
    \item Luego sumar y restar ese valor de $B$ averiguado. Como se lo suma y resta es como sumar $0$, lo que es matemáticamente correcto.
    \item Agrupar los términos del trinomio $A^2 + 2AB + B^2$ y transformarlo en un binomio ($(A+B)^2$)
    \item Resolver el ejercicio
\end{enumerate}

\subsection{Descomposición en fracciones parciales}