\section{Límites}

Para que exista el límite este tiene que ser igual por izquierda y por derecha. Es único. 

\subsection{Indeterminaciones}

Es muy frecuente que al resolver un ejercicio de límites nos encontremos con indeterminaciones, casos que no se pueden resolver directamente. Algunos escenarios son:

\begin{multicols}{2}
\begin{itemize}
\item $\infty + \infty = \infty$

\item $\infty - \infty = ???$

\item $\infty \cdot \infty = \infty$

\item $\dfrac{\infty}{\infty} = ???$

\item $\infty + k = \infty, \;\;\;k\in\R$

\item $\infty \cdot k = \infty, \;\;\;k\in\R>0$

\item $\infty \cdot k = -\infty, \;\;\;k\in\R<0$

\item $\infty \cdot 0 = 0, \;\;\;\text{Si 0 es el número}$

\item $\infty \cdot 0 = ???, \;\;\;\text{Si 0 es un límite}$

\item $\dfrac{\infty}{k}= \infty, \;\;\;k\in\R^+$

\item $\dfrac{\infty }{k} = -\infty, \;\;\;k\in\R^-$

\item $\dfrac{\infty }{0} = \infty$

\item $\dfrac{k}{\infty} = 0$

\item $\dfrac{k}{0} = \infty$

\item $\dfrac{0}{0} = ???$

\item $\ln (\infty) = \infty$

\item $k^\infty = \infty, \;\;\;k\in\R>1$

\item $1^\infty = 1$

\item $k^\infty = 0, \;\;\;k\in\R,\; 0<k<1$

\item $0^\infty = 0$

\item $k^\infty =  \text{No existe}, \;\;\;k\in\R^-$
\item[]
\end{itemize}
\end{multicols}

\subsection{Límites trigonométricos (con l'Hôpital no hace falta recordar esto)}

\hfil
$\lim\limits_{x\rightarrow 0} \dfrac{\sin(x)}{x} = 1$
\hfil
$\lim\limits_{x\rightarrow 0} \dfrac{1 - \cos(x)}{x} = 0$
\hfil

\subsection{L'Hôpital}

Donde $\otimes$ es cualquier número o infinito que da una indeterminación 0/0 o $\infty/\infty$.

$$\lim\limits_{x\rightarrow \otimes}\dfrac{f(x)}{g(x)}=
\lim\limits_{x\rightarrow \otimes}\dfrac{f'(x)}{g'(x)}$$
