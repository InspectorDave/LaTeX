\section{Funciones multivariable}

\subsection*{Cálculo de máximos y mínimos}

Se utiliza el Hessiano. El hessiano es el determinante de la siguiente matriz, en la que se analizan los PUNTOS CRÍTICOS. Para averiguar los puntos críticos, evaluar cuándo $f'_{x}(x,y) = 0$ y $f'_{y}(x,y)=0$ (todos los puntos que resuelvan el sistema de ecuaciones serán puntos críticos.

{
\Large
\renewcommand{\arraystretch}{1.4}
\[
H(x_0;y_o) =
\left|
\begin{array}{c@{\hspace{5mm}}c}
A & B\\
B & C\\
\end{array}
\right|
=
\left|
\begin{array}{c@{\hspace{8mm}}c}
f_{xx}^{''}(x_0;y_0) & f_{xy}^{''}(x_0;y_0)\\
f_{xy}^{''}(x_0; y_0) & f_{yy}^{''}(x_0; y_0)\\
\end{array}\right|
\]
}

Según los valores del determinante:
\begin{itemize}
\item $H(x,y) > 0$: Hay extremo
\item $H(x,y) < 0$: No hay
\item $H(x,y) = 0$: No se sabe
\end{itemize}

Si $f''_{xx}(x_0,y_0)$ y $f''_{yy}(x_0,y_0)$ ambas son mayores positivas es un mínimo, si ambas son negativas es un máximo (con analizar una sola alcanza). Si tienen distinto signo, punto ensilladura.