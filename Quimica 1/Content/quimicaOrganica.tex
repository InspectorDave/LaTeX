\section{Química orgánica}

Fórmula desarrollada, semidesarrollada, taquigráfica y condensada.


\subsection*{Hidrocarburos}

Son compuestos moleculares que solo tienen carbono e hidrógeno. Los alcanos tienen solo enlaces covalentes simples, los alquenos al menos un enlace covalente doble y los alquinos al menos un enlace covalente triple.

\begin{enumerate}
    \item Identificar la cadena más larga.
    \item Identificar el extremo inicial (el que esté más cerca del grupo funcional o del grupo más largo).
    \item En orden alfabético, nombrar en qué posición está cada grupo (repitiendo posiciones repetidas), sin considerar el di, tri, tetra.
    \item Agregar el nombre de la cadena + el sufijo.
\end{enumerate}


\subsubsection*{Alcanos}

Se nombran con el prefijo dependiendo de cuántos carbonos tenga (meta, eta, propa, buta, penta, hexa, hepta, octa, nona, deca...)  más ``ano'' al final. Una vez establecida la cantidad de carbonos saturar el resto de los enlaces con hidrógenos. Por ejemplo CH$_4$ es metano, C$_2$H$_6$ es etano, C$_3$H$_8$ es propano, C$_4$H$_{10}$ es butano y así. La relación entre carbonos e hidrógenos es $H = 2C + 2$.

En caso que la molécula no sea lineal (ni cíclica), se nombra como el largo de la cadena principal más los nombres de cada ramificación.

Primero se ubican las posiciones de las ramificaciones de largo 1 con respecto a la cadena más larga, luego las de largo 2, luego largo 3 y así sucesivamente. La posición será desde el extremo que genere las posiciones más chicas.

Luego se escribe cada posición donde haya una remificación, en caso de haber dos ramificaciones se escribe esa posición dos veces.

Por ejemplo: 3etil-2,2-dimetil octano.


\subsubsection*{Alquenos}

La cadena principal será la más larga que contenga al enlace doble. Se numera desde el extremo más cercano al enlace doble. Se agrega un número diciendo dónde está el enlace triple.


\subsubsection*{Alquinos}

La cadena principal será la más larga que contenga al enlace triple. Se numera desde el extremo más cercano al enlace triple. Se agrega un número diciendo dónde está el enlace triple.


\subsection*{Alcoholes}

Tiene que haber un OH en la cadena principal, se nombran metanol, etanol y así. Se pone en qué posición de la cadena principal está el grupo OH.

En caso de ser alcoholes dobles, se nombran como 1,3-butano


\subsection*{Ácidos carboxílicos}

Hay un carbono enlazado a un OH  ya un O con enlace doble en algún extremo de la cadena. Se nombran agregando la terminación -oico. Metanoico, propanoico, etc.


\subsection*{Cetonas}

Hay un C con un doble enlace a un O (grupo carbonilo), enlazado a otros 2 carbonos. Terminan con anal. Propanona, pentanona.


\subsection*{Aldehídos}

Hay un grupo funcional carbonilo, carbono enlazado a un O con enlace doble y a un H. Propanal, pentanal, etc.


\section*{Isomería}

Son moléculas con la misma fórmula química pero distinta estructura.


\subsubsection*{Isomería de cadena}

Cambian las posiciones de los carbonos e hidrógenos. El pentano es isómero de cadena del 2-metil-butano por ejemplo.


\subsubsection*{Isómero de posición}

Lo único que cambia es la posición del enlace doble, triple o del grupo funcional. Por ejemplo, el 1-penteno es isómero de posición del 2-penteno.


\subsubsection*{Isómero de función}

Son dos moléculas con igual fórmula molecular salvo por el tipo de grupo funcional que los distingue. Por ejemplo 