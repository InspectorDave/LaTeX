\section{Balanceo de ecuaciones}

\subsection{Método algebraico}

Cada elemento genera una ecuación. A cada compuesto se le asigna una letra (incógnita). Luego en cada ecuación se multiplica cada letra por la cantidad de veces que aparece el elemento en el compuesto y finalmente se resuelve el sistema de ecuaciones.

\skipline
\underline{Ejemplo:}

$$\text{Fe} + \text{HCl} \longrightarrow \text{FeCl}_3 + \text{H}_2$$

\begin{minipage}[l]{0.2\textwidth}
\begin{align*}
\text{Fe}&: A = C\\
\text{H}&: B = 2D\\
\text{Cl}&: B = 3C\\
\;
\end{align*}
\end{minipage}
$\Longrightarrow$
\hfil
$B=2D=3C=3A$
\hfil
$\Longrightarrow$
\begin{minipage}[l]{0.1\textwidth}
\begin{align*}
A&= 2\\
B&= 6\\
C&= 2\\
D&= 3
\end{align*}
\end{minipage}
$\Longrightarrow$
\hfil 
\fbox{$2\text{Fe} + 6\text{HCl} \longrightarrow 2\text{FeCl}_3 + 3\text{H}_2$}
\hfil

\subsection{Redox}

\subsubsection*{N$^\circ$ de oxidación}

Es la cantidad de electrones que un compuesto pone en juego.

Elementos pareados consigo mismo: 0. Elementos en estado elemental: 0. Iones: su carga. Oxígeno: -2. Hidrógeno: -1/+1. Alcalinos: +1. Alcalinotérreos: +2. Los no metales cuando se ponen con un metal tienen su número de oxidación negativo.

Entre dos elementos, el más electronegativo es que tendrá número de oxidación negativo.

\subsubsection*{Reacción redox}

Es una reacción química donde hay transferencia de electrones (cambian los números de oxidación).

El reductor es el que pierde electrones, se oxida, el oxidante es el que \underline{re}cibe electrones, se \underline{re}duce.

Para resolver un ejercicio de redox:
\begin{enumerate}[itemsep=0pt, parsep=0.3em, topsep=0.3em]
    \item Identificar al compuesto que se oxida y al que se reduce.
    \item Identificar si se está en medio ácido o básico (puedo no estarse en ninguno).
    \item Escribir la semirreacción de oxidación y la de reducción, poniendo solo los iones afectados.
    \item Balancear compuestos y cargas (agregando los $e^-$ y los H$^+$, (OH)$^-$ según el medio. En medio básico, las aguas van donde hay más oxígenos. Agregar un agua por cada oxígeno que falte y el doble de OH.
    \item Poner todos los electrones en el mismo lado de las ecuaciones, y multiplicar de manera que queden la misma cantidad en cada una.
    \item Sumar las ecuaciones resultantes, se tienen que cancelar los electrones y luego simplificar si se da el caso.
    \item Finalmente llevar los coeficientes obtenidos a la fórmula original, agregando por tanteo lo que haga falta.
\end{enumerate}

\newpage
\textbf{\underline{Ejemplo simple:}}

$$\text{Fe} + \text{Cl}_2 \longrightarrow \text{FeCl}_2$$

\begin{multicols}{2}

\underline{Semirreacción de oxidación:}

$$\text{Fe} - 2e^- \longrightarrow \text{Fe}^{2+}$$

\underline{Semirreacción de reducción:}

$$\text{Cl}_2 + 2e^- \longrightarrow 2\text{Cl}^{-} $$

\end{multicols}

Las reacciones Redox pueden ser en medios ácidos o básicos. En caso de ser medio ácido, hay que compensar la falta de hidrógenos con cationes H$^+$. En caso de ser medio básico esta falta se compensa con grupos hidroxilos (OH)$^-$.

\skipline
\textbf{\underline{Ejemplo medio ácido:}}

$$\text{KI} + \text{K} \text{I} \text{O}_3 + \text{H} \text{Cl} \longrightarrow
\text{KCl} + \text{I}_2 +  \text{H}_2 \text{O}$$

\hfil Ioduro de potasio \hfil + \hfil iodato de potasio \hfil + \hfil ácido clorhídrico \hfil
$\rightarrow$ \hfil
cloruro de potasio \hfil + \hfil iodo molecular \hfil + \hfil agua \hfil 

\begin{multicols}{2}

\underline{Semirreacción de oxidación:}

$$2\text{I}^{-} \longrightarrow \text{I}_2 + 2e^-$$

\underline{Semirreacción de reducción:}

$$2(\text{IO}_3)^{-} + 12 \text{H}^+ \longrightarrow \text{I}_2 + 6\text{H}_2 \text{O} - 10e^-$$
\end{multicols}

Multiplico para que la cantidad de electrones sean iguales (multiplicar por cinco la de oxidación) y las sumo:

$$10\text{I}^{-} + 2(\text{IO}_3)^{-} + 12 \text{H}^+ \longrightarrow 5 \text{I}_2 + 10 e^- + \text{I}_2 + 6\text{H}_2 \text{O} - 10e^-$$

Cancelo los electrones, y si puedo simplifico, quedando:

$$5\text{I}^{-} + (\text{IO}_3)^{-} + 6 \text{H}^+ \longrightarrow 3 \text{I}_2 + 3\text{H}_2 \text{O}$$

Finalmente pongo los coeficientes en la fórmula original y agrego lo que haga falta:

$$5\text{KI} + \text{K} \text{I} \text{O}_3 + 6\text{H} \text{Cl} \longrightarrow
6\text{KCl} + 3\text{I}_2 +  3\text{H}_2 \text{O}$$
