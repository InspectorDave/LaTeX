\section{Soluciones}

Una solución es una mezcla homogenea entre dos o más sustancias.

Solución (SC): es un compuesto en el que todas sus partes tienen las mismas propiedades intrínsecas. Está formada por un soluto y un solvente (puede haber más de un soluto). Si tengo agua salada, el agua salada es la solución.
    
Solvente (SV): es el componente de una solución que disuelve al soluto. Si tengo agua salada, el solvente es el agua.
    
Soluto (ST): es lo que está disuelto en la solución. Si tengo agua salada, el soluto es la sal.

\subsubsection*{Densidad}

Es la relación entre la masa de un compuesto y su volumen.

$$\delta = \dfrac{m}{V}$$

\subsubsection*{Concentración de soluciones}

\vspace{-10mm}
\begin{multicols}{3}
\begin{align*}
    \% m/m &= \dfrac{ m_{st}}{ m_{sc}} \cdot 100\%\\
    \% V/V &= \dfrac{ V_{st}}{ V_{sc}} \cdot 100\%\\
    \% m/V &= \dfrac{ m_{st} \text{ (g)}}{ V_{sc} \text{ (ml)}} \cdot 100\%
\end{align*}   

\begin{align*}
    M &= \dfrac{n_{st}}{V_{sc} \text{ (l)}}\\
    m &= \dfrac{n_{st}}{m_{sv}  \text{ (kg)}}
\end{align*}

\begin{align*}
    \text{ppm} &= \dfrac{m_{st} \text{ (mg)}}{ m_{sc}  \text{ (kg)}} \;\;\text{ ó }\;\; \dfrac{m_{st} \text{ (mg)}}{ V_{sc} \text{ (l)} }\\
    X_{st} &= \dfrac{n_{st}}{n_{st} + n_{sv}}\\
    X_{sv} &= \dfrac{n_{sv}}{n_{st} + n_{sv}}
\end{align*}
    
\end{multicols}

\subsubsection*{pH}

Representa la molaridad de $H^+$ que hay en una solución.

\hfil$\text{pH} = -\log [\text{H}^+]$\hfil

Tener en cuenta que $\text{pH} + \text{pOH} = 14$. El pH del agua es 7. Si el pH es menor a 7 es ácido, si es mayor a 7 es básico. Si una solución supuestamente ácida da básica, o básica da ácida, ese pH no está bien, se descarta porque empieza a tener más importancia la ionización del agua y el pH final es 7.


