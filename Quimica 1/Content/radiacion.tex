
\section{Química nuclear}

\subsection*{Fusión}

Es un proceso mediante el cual el núcleo de un átomo crece, ya sea porque se le agregan neutrones, protones o porque se fusiona con otros átomos.

Es un proceso que requiere mucha energía, en la naturaleza ocurre en las estrellas y en las capas altas de la atmósfera (esto último es más raro).

\subsection*{Fisión}

Es un proceso mediante el cual el núcleo de un átomo decrece. Cuando un átomo emite radiación alfa o beta, está ocurriendo fisión. 

Este proceso ocurre naturalmente en los minerales radiactivos. Artificialmente ocurre en reactores nucleares o bombas nucleares.

\subsection*{Radiación}

Algunos isótopos de determinados elementos no son estables. Estos átomos inestables emiten radiación. Esta radiación puede ser de 3 tipos:

\subsubsection*{Radiación alfa($\alpha$)}

Cuando un átomo emite radiación alfa, este desprende 2 protones y 2 neutrones, se puede decir que desprende un átomo de helio que no tiene electrones.

\vspace{0.2cm}
La radiación alfa es posible de frenar con una hoja de papel. Es muy ionizante (cuanto más ionizante más probabilidad que te cause cáncer), así que es muy peligroso respirar o tragar este tipo de partículas.

\subsubsection*{Radiación beta ($\beta$)}

Cuando un átomo emite radiación beta, este desprende un electrón. Además uno de sus neutrones se transformará en un protón (modificando el número atómico del átomo pero no el número másico). 

\vspace{0.2cm}
Existen casos particulares en los que en vez de desprenderse un electrón se desprende un positrón (una partícula flashera, es como un electrón con carga positiva). Y en vez de que un neutrón se transforme en un protón pasa lo opuesto: un protón se transforma en un neutrón. Cuando pasa esto se habla de una emisión beta más ($\beta^+$).

\vspace{0.2cm}
Una manera de frenar la radiación gamma es con planchas de aluminio. Es menos ionizante que las partículas alfa.

\subsubsection*{Radiación gamma $(\gamma)$}

Este tipo de radiación ocurre cuando un átomo emite energía electromagnética en forma de un fotón (no emite materia, solo emite energía).

Es la menos ionizante de las tres radiaciones, pero al ser energía es muy difícil de frenar. Para frenarla se utilizan placan de plomo, hormigón o agua.

\subsubsection*{Vida media}

La vida media de un material es la cantidad de tiempo que tiene que pasar para que una concentración de dicho material se descomponga a la mitad. La ecuación para saber cuánto material se tendrá luego de una cantidad de tiempo es ($m_f$ es masa final, $m_0$ es masa inicial, $e$ es el número irracional, $t$ el tiempo que pasará y $V$ la vida media. A veces puede aparecer la constante de desintegración $\lambda=\ln(2)/V$):

\vspace{3mm}
\hfil{\LARGE
$m_f = m_0 \cdot e^{\frac{-t\cdot \ln(2)}{V}}=
m_0 \cdot 2^{\frac{-t}{V}}$\hfil
}